%% LyX 2.0.6 created this file.  For more info, see http://www.lyx.org/.
%% Do not edit unless you really know what you are doing.
\documentclass[english,hyperref]{beamer}
\usepackage[latin9]{inputenc}
\setcounter{secnumdepth}{3}
\setcounter{tocdepth}{3}
\usepackage{babel}
\usepackage{amsmath}
\usepackage{amssymb}
\ifx\hypersetup\undefined
  \AtBeginDocument{%
    \hypersetup{unicode=true,pdfusetitle,
 bookmarks=true,bookmarksnumbered=true,bookmarksopen=false,
 breaklinks=false,pdfborder={0 0 1},backref=false,colorlinks=true,
 linkcolor=black,urlcolor=blue}
  }
\else
  \hypersetup{unicode=true,pdfusetitle,
 bookmarks=true,bookmarksnumbered=true,bookmarksopen=false,
 breaklinks=false,pdfborder={0 0 1},backref=false,colorlinks=true,
 linkcolor=black,urlcolor=blue}
\fi

\makeatletter

%%%%%%%%%%%%%%%%%%%%%%%%%%%%%% LyX specific LaTeX commands.
%% Because html converters don't know tabularnewline
\providecommand{\tabularnewline}{\\}

%%%%%%%%%%%%%%%%%%%%%%%%%%%%%% Textclass specific LaTeX commands.
 % this default might be overridden by plain title style
 \newcommand\makebeamertitle{\frame{\maketitle}}%
 \AtBeginDocument{
   \let\origtableofcontents=\tableofcontents
   \def\tableofcontents{\@ifnextchar[{\origtableofcontents}{\gobbletableofcontents}}
   \def\gobbletableofcontents#1{\origtableofcontents}
 }
 \long\def\lyxframe#1{\@lyxframe#1\@lyxframestop}%
 \def\@lyxframe{\@ifnextchar<{\@@lyxframe}{\@@lyxframe<*>}}%
 \def\@@lyxframe<#1>{\@ifnextchar[{\@@@lyxframe<#1>}{\@@@lyxframe<#1>[]}}
 \def\@@@lyxframe<#1>[{\@ifnextchar<{\@@@@@lyxframe<#1>[}{\@@@@lyxframe<#1>[<*>][}}
 \def\@@@@@lyxframe<#1>[#2]{\@ifnextchar[{\@@@@lyxframe<#1>[#2]}{\@@@@lyxframe<#1>[#2][]}}
 \long\def\@@@@lyxframe<#1>[#2][#3]#4\@lyxframestop#5\lyxframeend{%
   \frame<#1>[#2][#3]{\frametitle{#4}#5}}
 \def\lyxframeend{} % In case there is a superfluous frame end
 \newenvironment{lyxcode}
   {\par\begin{list}{}{
     \setlength{\rightmargin}{\leftmargin}
     \setlength{\listparindent}{0pt}% needed for AMS classes
     \raggedright
     \setlength{\itemsep}{0pt}
     \setlength{\parsep}{0pt}
     \normalfont\ttfamily}%
    \def\{{\char`\{}
    \def\}{\char`\}}
    \def\textasciitilde{\char`\~}
    \item[]}
   {\end{list}}

%%%%%%%%%%%%%%%%%%%%%%%%%%%%%% User specified LaTeX commands.
\usetheme[secheader]{Boadilla}
\usecolortheme{seahorse}
\title[Elm-style FRP]{Elm-style Functional Reactive Programming demystified}
\author{Sergei Winitzki}
\date{April 13, 2015}
\institute[Versal Group Inc.]{SF Types, Theorems, and Programming Languages}

\makeatother

\begin{document}
\frame{\titlepage}


\lyxframeend{}\lyxframe{Part 1. Functional reactive programming in \texttt{Elm}}

FRP has little to do with...
\begin{itemize}
\item multithreading, message-passing concurrency, ``actors''
\item distributed computing on massively parallel, load-balanced clusters
\item map/reduce, the ``reactive manifesto''
\end{itemize}
FRP means...
\begin{itemize}
\item \textbf{pure functions }using\textbf{ temporal types} as primitives

\begin{itemize}
\item (temporal type $\approx$ lazy stream of events)
\end{itemize}
\end{itemize}
FRP is probably most useful for:
\begin{itemize}
\item GUI programming
\end{itemize}
\texttt{Elm} is...
\begin{itemize}
\item a viable implementation of FRP geared for Web apps
\end{itemize}

\lyxframeend{}


\lyxframeend{}\lyxframe{Transformational vs. reactive programs}

\begin{center}
\begin{tabular}{|c|c|}
\hline 
\textbf{Transformational} programs & \textbf{Reactive} programs\tabularnewline
\hline 
\hline 
\textbf{example}: \texttt{\textcolor{blue}{\footnotesize{pdflatex
elm\_talk.tex}}} & \textbf{example}: any GUI program, OS\tabularnewline
\hline 
start, run, then stop & keep running indefinitely\tabularnewline
\hline 
read some input, write some output & wait for signals, send messages \tabularnewline
\hline 
\textbf{execution:} sequential, parallel & ``main run loop'' + concurrency\tabularnewline
\hline 
\textbf{difficulty:} algorithms & signal/response sequences\tabularnewline
\hline 
\textbf{specification:} classical logic? & classical temporal logic?\tabularnewline
\hline 
\textbf{verification:} proof of correctness? & model checking?\tabularnewline
\hline 
\textbf{synthesis:} extract code from proof? & temporal logic synthesis? \tabularnewline
\hline 
\textbf{type theory:} intuitionistic logic & intuitionistic \emph{temporal} logic\tabularnewline
\hline 
\end{tabular}
\par\end{center}


\lyxframeend{}


\lyxframeend{}\lyxframe{Difficulties in reactive programming}

Usually, reactive programs are written imperatively...
\begin{itemize}
\item Input signals may come at unpredictable times

\begin{itemize}
\item Imperative updates are difficult to keep in the correct order
\item Flow of events becomes difficult to understand
\end{itemize}
\item Asynchronous (out-of-order) callback logic becomes opaque

\begin{itemize}
\item ``callback hell'': deeply nested callbacks, all mutating data
\end{itemize}
\item Inverted control (``the system will call you'') obscures the flow
of data
\item Some concurrency is usually required (e.g.~background tasks)

\begin{itemize}
\item Explicit multithreaded code is hard to write and debug
\end{itemize}
\end{itemize}

\lyxframeend{}


\lyxframeend{}\lyxframe{Motivation for FRP}
\begin{itemize}
\item Reactive programs work on \textbf{infinite sequences} of input/output
values
\item Main idea: make infinite sequences implicit, as a new ``\textbf{temporal}''
type

\begin{itemize}
\item (Elm) \texttt{Signal} $\alpha$ --- an infinite sequence of values
of type $\alpha$
\item alternatively, a value of type $\alpha$ that ``changes with time''
\end{itemize}
\item Reactive programs are \textbf{pure functions}

\begin{itemize}
\item a GUI is a pure function of type \texttt{Signal} \texttt{Inputs} $\rightarrow$
\texttt{Signal} \texttt{View} 
\item a Web server is a pure function \texttt{Signal} \texttt{Request} $\rightarrow$
\texttt{Signal} \texttt{Response} 
\item all mutation is \textbf{implicit} in \texttt{Signal} $\alpha$; our
code is 100\% immutable

\begin{itemize}
\item instead of updating an \texttt{x:Int}, we define a value of type \texttt{Signal}
\texttt{Int}
\end{itemize}
\item asynchronous behavior is \textbf{implicit}: our code has no callbacks
\item concurrency / parallelism is \textbf{implicit}

\begin{itemize}
\item the FRP runtime will provide the required scheduling of events
\end{itemize}
\end{itemize}
\end{itemize}

\lyxframeend{}


\lyxframeend{}\lyxframe{Czaplicki's original \texttt{Elm} 2012 in a nutshell}
\begin{itemize}
\item \texttt{Elm} is a pure polymorphic $\lambda$-calculus with products
and sums
\item \textbf{Temporal type} $\Sigma\alpha$ --- a lazy sequence of values
of type $\alpha$
\item Temporal \textbf{combinators} in core \texttt{Elm}: \end{itemize}
\begin{lyxcode}
constant:~$\alpha\rightarrow\Sigma\alpha$

map2:~$(\alpha\rightarrow\beta\rightarrow\gamma)\rightarrow\Sigma\alpha\rightarrow\Sigma\beta\rightarrow\Sigma\gamma$

foldp:~\textrm{$\left(\alpha\rightarrow\beta\rightarrow\beta\right)\rightarrow\beta\rightarrow\Sigma\alpha\rightarrow\Sigma\beta$}

async:~$\Sigma\alpha\rightarrow\Sigma\alpha$\end{lyxcode}
\begin{itemize}
\item \textbf{No nested} temporal types: \texttt{constant (constant x)}
is ill-typed!
\item Domain-specific primitive types: \texttt{Bool}, \texttt{Int}, \texttt{Float},
\texttt{String}, \texttt{View}
\item Standard library with data structures, HTML, HTTP, JSON, ...

\begin{itemize}
\item ...and signals \texttt{Time.every}, \texttt{Mouse.position}, \texttt{Window.dimensions},
...
\item ...and some utility functions: \texttt{map}, \texttt{merge}, \texttt{drop},
...
\end{itemize}
\end{itemize}

\lyxframeend{}


\lyxframeend{}\lyxframe{Details: \texttt{Elm} type judgments {[}Czaplicki 2012{]}}
\begin{itemize}
\item Polymorphically typed $\lambda$-calculus (also with temporal types)
\end{itemize}
\begin{align*}
\frac{\Gamma,(x:\alpha)\vdash e:\beta}{\Gamma\vdash(\lambda x.e):\alpha\rightarrow\beta}\,\textsc{Lambda} & \quad\frac{\Gamma\vdash e_{1}:\alpha\rightarrow\beta\quad\Gamma\vdash e_{2}:\alpha}{\Gamma\vdash(e_{1}\, e_{2}):\beta}\,\textsc{Apply}
\end{align*}

\begin{itemize}
\item Temporal types are denoted by $\Sigma\tau$

\begin{itemize}
\item In these rules, type variables $\alpha,\beta,\gamma$ \textbf{cannot}
involve $\Sigma$:
\end{itemize}
\end{itemize}
\begin{align*}
\frac{\Gamma\vdash e:\alpha}{\Gamma\vdash\left(\mbox{constant}\, e\right):\Sigma\alpha} & \:\textsc{Const}\\
\frac{\Gamma\vdash m:\alpha\rightarrow\beta\rightarrow\gamma\quad\Gamma\vdash p:\Sigma\alpha\quad\Gamma\vdash q:\Sigma\beta}{\Gamma\vdash\left(\mbox{map2}\: m\: p\: q\right):\Sigma\gamma} & \:\textsc{Map2}\\
\frac{\Gamma\vdash u:\alpha\rightarrow\beta\rightarrow\beta\quad\Gamma\vdash e:\beta\quad\Gamma\vdash q:\Sigma\alpha}{\Gamma\vdash\left(\mbox{foldp}\, u\, e\, q\right):\Sigma\beta} & \:\textsc{FoldP}
\end{align*}

\begin{itemize}
\item A value of type $\Sigma\Sigma\alpha$ is impossible in a well-typed
expression!
\end{itemize}

\lyxframeend{}


\lyxframeend{}\lyxframe{\texttt{Elm} operational semantics 1: Current values}
\begin{itemize}
\item Non-temporal expressions are evaluated \textbf{eagerly} in pure $\lambda$-calculus
\item Temporal expressions are built from \textbf{input} signals and combinators

\begin{itemize}
\item It is not possible to ``consume'' a signal ($\Sigma\alpha\rightarrow\beta$)!
\end{itemize}
\item Every temporal expression has a \textbf{current value} denoted by
$e^{\left[c\right]}$
\end{itemize}
\[
\frac{\Gamma\vdash e:\Sigma\alpha\quad\quad\Gamma\vdash c:\alpha}{\Gamma\vdash e^{\left[c\right]}:\Sigma\alpha}\:\textsc{CurVal}
\]

\begin{itemize}
\item Every predefined \textbf{input} signal $i:\Sigma\alpha,i\in{\cal I}$
has an initial value: $i^{[a]}$
\item \textbf{Initial} current values for all expressions are derived:
\end{itemize}
\begin{align*}
\frac{\Gamma\vdash\left(\mbox{constant}\ c\right):\Sigma\alpha}{\Gamma\vdash\left(\mbox{constant}\ c\right)^{\left[c\right]}:\Sigma\alpha} & \:\textsc{ConstInit}\\
\frac{\Gamma\vdash\left(\mbox{map2}\ m\ p^{\left[a\right]}\ q^{\left[b\right]}\right):\Sigma\gamma}{\Gamma\vdash\left(\mbox{map2}\ m\ p\ q\right)^{\left[m\ a\ b\right]}:\Sigma\gamma} & \:\textsc{Map2Init}\\
\frac{\Gamma\vdash\left(\mbox{foldp}\, u\, e\, q\right):\Sigma\beta}{\Gamma\vdash\left(\mbox{foldp}\, u\, e\, q\right){}^{\left[e\right]}:\Sigma\beta} & \:\textsc{FoldPInit}
\end{align*}



\lyxframeend{}


\lyxframeend{}\lyxframe{\texttt{Elm} operational semantics 2: Update steps}
\begin{itemize}
\item Update steps happen only to \textbf{input signals} $s\in{\cal I}$
and \textbf{one at a time}
\item Update steps $\mathbf{U}_{s\leftarrow a}\left\{ ...\right\} $ are
applied to the \textbf{whole program} at once:
\end{itemize}
\[
\frac{\Gamma\vdash s:\Sigma\alpha\quad s\in{\cal I}\quad\Gamma\vdash a:\alpha\quad\Gamma\vdash e^{\left[c\right]}:\Sigma\beta\quad\Gamma\vdash e'^{\left[c'\right]}:\Sigma\beta}{\Gamma\vdash\mathbf{U}_{s\leftarrow a}\left\{ e^{\left[c\right]}\right\} \Rightarrow e'^{\left[c'\right]}}
\]

\begin{itemize}
\item An update step on $s$ will leave all other \textbf{input} signals
unchanged:
\end{itemize}
\[
\forall s\neq s'\in{\cal I}:\quad\quad\mathbf{U}_{s\leftarrow b}\left\{ s^{\left[a\right]}\right\} \Rightarrow s^{\left[b\right]}\quad\quad\mathbf{U}_{s\leftarrow b}\left\{ s'^{\left[c\right]}\right\} \Rightarrow s'^{\left[c\right]}
\]

\begin{itemize}
\item Efficient implementation:

\begin{itemize}
\item The instances of input signals within expressions are not duplicated
\item Unchanged current values are cached and not recomputed
\end{itemize}
\end{itemize}

\lyxframeend{}\lyxframe{\texttt{Elm} operational semantics 3: Updating combinators}
\begin{itemize}
\item Operational semantics does not reduce temporal expressions

\begin{itemize}
\item The whole program \textbf{remains} a static temporal expression tree
\item Only the current values are updated in all subexpressions
\end{itemize}
\end{itemize}
\begin{align*}
 & \mathbf{U}_{s\leftarrow a}\left\{ \left(\mbox{constant}\ c\right)^{\left[c\right]}\right\} \Rightarrow\left(\mbox{constant}\ c\right)^{\left[c\right]} & \textsc{ConstUpd}\\
 & \mathbf{U}_{s\leftarrow a}\left\{ \mbox{map2}\ m\ p\ q\right\} \\
 & \quad\quad\Rightarrow\left(\mbox{map2}\ m\ \mathbf{U}_{s\leftarrow a}\left\{ p\right\} ^{\left[b\right]}\ \mathbf{U}_{s\leftarrow a}\left\{ q\right\} ^{\left[c\right]}\right)^{\left[m\ b\ c\right]} & \textsc{Map2Upd}\\
 & \mathbf{U}_{s\leftarrow a}\left\{ \left(\mbox{foldp}\, u\, e\, q\right)^{\left[b\right]}\right\} \Rightarrow\left(\mbox{foldp}\, u\, e\,\mathbf{U}_{s\leftarrow a}\left\{ q\right\} ^{\left[c\right]}\right){}^{\left[u\ c\ b\right]} & \textsc{FoldPUpd}
\end{align*}

\begin{itemize}
\item All computations during an update step are \textbf{synchronous}

\begin{itemize}
\item The expression $\mathbf{U}_{s\leftarrow b}\left\{ e^{\left[c\right]}\right\} $
is reduced only after all subexpressions of $e$
\end{itemize}
\end{itemize}

\lyxframeend{}


\lyxframeend{}\lyxframe{GUI building: ``Hello, world'' in \texttt{Elm} }
\begin{itemize}
\item The value called \texttt{\textbf{main}} will be visualized by the
runtime\end{itemize}
\begin{lyxcode}
import~Graphics.Element~(..)~

import~Text~(..)~

import~Signal~(..)

~~

text~:~Element

text~=~plainText~\char`\"{}Hello,~World!\char`\"{}

~~

main~:~Signal~Element~

main~=~constant~text\end{lyxcode}
\begin{itemize}
\item Try \texttt{Elm} online at \href{http://elm-lang.org/try}{http://elm-lang.org/try}
\end{itemize}

\lyxframeend{}


\lyxframeend{}\lyxframe{Example of using \texttt{foldp} }
\begin{itemize}
\item Specification:

\begin{itemize}
\item \textcolor{blue}{\emph{I work only after the boss comes by and unless
the phone rings}} 
\end{itemize}
\item Implementation:\end{itemize}
\begin{lyxcode}
after\_unless~:~(Bool,~Bool)~-\textgreater{}~Bool~-\textgreater{}~Bool

after\_unless~(b,r)~w~=~(w~\textbar{}\textbar{}~b)~\&\&~not~r

~$\quad\quad$

boss~:~Signal~Bool

phone:~Signal~Bool

~$\quad\quad$

i\_work~:~Signal~Bool

i\_work~=~foldp~after\_unless~false~(boss,~phone)\end{lyxcode}
\begin{itemize}
\item Demo
\end{itemize}

\lyxframeend{}


\lyxframeend{}\lyxframe{Typical GUI boilerplate in \texttt{Elm} }
\begin{itemize}
\item A state machine with stepwise update:\end{itemize}
\begin{lyxcode}
update~:~Command~$\rightarrow$~State~$\rightarrow$~State\end{lyxcode}
\begin{itemize}
\item A rendering function:\end{itemize}
\begin{lyxcode}
draw~:~State~$\rightarrow$~View\end{lyxcode}
\begin{itemize}
\item A manager that merges the required input signals into one:

\begin{itemize}
\item may use \texttt{Mouse}, \texttt{Keyboard}, \texttt{Time}, \texttt{HTML}
stuff, etc.
\end{itemize}
\end{itemize}
\begin{lyxcode}
merge\_inputs~:~Signal~Command\end{lyxcode}
\begin{itemize}
\item Main boilerplate:\end{itemize}
\begin{lyxcode}
init\_state~:~State

main~:~Signal~View

main~=~map~draw~(foldp~update~init\_state~merge\_inputs)
\end{lyxcode}

\lyxframeend{}


\lyxframeend{}\lyxframe{Asynchrony and concurrency in \texttt{Elm}}
\begin{itemize}
\item Long-running computations will delay signal updates!
\item Example: \texttt{map f s} where \texttt{f:$\alpha\rightarrow\alpha$}
takes a long time to compute
\item Use \texttt{async} \texttt{:} $\Sigma\alpha\rightarrow\Sigma\alpha$
\item Operational semantics: ($i$ is a \textbf{new} input signal for each
\texttt{async})
\begin{align*}
\frac{\Gamma\vdash e^{\left[c\right]}:\Sigma\alpha}{\Gamma,\left(i:\Sigma\alpha\right)\vdash\left(\mbox{async}_{i}\ e\right)^{\left[c\right]}:\Sigma\alpha} & \:\textsc{AsyncInit}\\
\mathbf{U}_{s\leftarrow a}\left\{ \left(\mbox{async}_{i}\ e\right)^{\left[c\right]}\right\} \Rightarrow\mathbf{U}_{i\leftarrow c'}^{\dagger}\left(\mbox{async}_{i}\ \mathbf{U}_{s\leftarrow a}^{\dagger}\left\{ e\right\} ^{\left[c'\right]}\right)^{\left[c\right]} & \:\textsc{AsyncSched}\\
\mathbf{U}_{i\leftarrow c'}\left\{ \left(\mbox{async}_{i}\ e\right)^{\left[c\right]}\right\} \Rightarrow\left(\mbox{async}_{i}\ e\right)^{\left[c'\right]} & \:\textsc{AsyncUpd}
\end{align*}

\item The update computation $\mathbf{U}_{s\leftarrow a}^{\dagger}\left\{ e\right\} $
runs on another thread...

\begin{itemize}
\item ...while the current value $c$ remains unchanged
\item Another update $\mathbf{U}_{i\leftarrow c'}^{\dagger}$ is \textbf{scheduled}
but not yet triggered
\item When $c'$ is ready, $\mathbf{U}_{i\leftarrow c'}\left\{ ...\right\} $
runs and sets the current value to $c'$
\end{itemize}
\end{itemize}

\lyxframeend{}


\lyxframeend{}\lyxframe{Example of using \texttt{async}}
\begin{itemize}
\item UI that shows results of some long computations:\end{itemize}
\begin{lyxcode}
draw~:~Int~-\textgreater{}~Int~-\textgreater{}~View

draw~x~y~=~...

~

s~:~Signal~Int~-{}-~input~values

~

fSlow~:~Int~-\textgreater{}~Int

res1~=~async~(map~fSlow~s)

fFast~:~Int~-\textgreater{}~Int

res2~=~map~fFast~s

~

main~:~Signal~View~=~map2~draw~res1~res2
\end{lyxcode}

\lyxframeend{}


\lyxframeend{}\lyxframe{Some limitations of \texttt{Elm}-style FRP}
\begin{itemize}
\item No higher-order signals: $\Sigma(\Sigma\alpha)$ is disallowed by
the type system
\item No distinction between continuous time and discrete time
\item The signal processing logic is fully specified statically
\item No constructors for user-defined signals
\item No recursion possible in signal definition!
\item Incomplete semantics for \texttt{async}\textrm{ : $\Sigma\alpha\rightarrow\Sigma\alpha$}

\begin{itemize}
\item Example: \texttt{async} \texttt{(map} \texttt{f} \texttt{s)} where
\texttt{f} takes a long time
\item The initial value of this signal will not be available at initial
time!
\item Need \texttt{async'} : $\alpha\rightarrow\Sigma\alpha\rightarrow\Sigma\alpha$
to specify initial value?
\end{itemize}
\item No full concurrency (e.g., ``dining philosophers'')
\end{itemize}

\lyxframeend{}


\lyxframeend{}\lyxframe{\texttt{Elm} cannot simulate ``dining philosophers''}
\begin{itemize}
\item A philosopher thinks for a random time, then eats for a random time

\begin{itemize}
\item Can a signal value \texttt{p} \texttt{:} \texttt{Signal} \texttt{Unit}
update itself at random times? 
\end{itemize}
\item No! There is no way to delay the update times of a signal \textbf{at
runtime}
\item \texttt{Time.delay:} \texttt{Int}$\rightarrow\Sigma\alpha\rightarrow\Sigma\alpha$
cannot use a time-varying delay value 
\item \texttt{Time.every:} \texttt{Int}$\rightarrow\Sigma$\texttt{Int}
also requires a fixed delay value
\item Cannot lift \texttt{Time.every} into \texttt{$\Sigma$Int}$\rightarrow\Sigma\Sigma$\texttt{Int}
to achieve variable delay
\end{itemize}

\lyxframeend{}


\lyxframeend{}\lyxframe{The JavaScript backend for \texttt{Elm} (2015)}

Features:
\begin{itemize}
\item Good support for HTML/CSS, HTTP requests, JSON
\item Good performance of caching HTML views
\item Support for Canvas and HTML-free UI building
\end{itemize}
Limitations:
\begin{itemize}
\item No implementation for \texttt{async} (JavaScript lacks concurrency)
\item The lack of recursive signals is compensated by \emph{ad hoc} primitives
\item Ordinary recursion may generate invalid JavaScript!
\end{itemize}

\lyxframeend{}


\lyxframeend{}\lyxframe{\texttt{Elm}-style FRP: the good parts}
\begin{itemize}
\item Transparent, declarative modeling of data through ADTs
\item Immutable and safe data structures (\texttt{Array}, \texttt{Dict},
...) 
\item No runtime errors or exceptions!
\item Space/time leaks are impossible!
\item Language is Haskell-like but simpler for beginners
\item Full type inference
\item Easy deployment and interop in Web applications
\end{itemize}

\lyxframeend{}


\lyxframeend{}\lyxframe{Some conservative extensions of \texttt{Elm}}
\begin{itemize}
\item Fix initial value semantics for \texttt{async'} \texttt{:} $\alpha\rightarrow\Sigma\alpha\rightarrow\Sigma\alpha$ 
\item Allow recursive definitions for signals

\begin{itemize}
\item generate updates as \texttt{s0}, \texttt{f(s0)}, \texttt{f(f(s0))},
... are being computed:\end{itemize}
\begin{lyxcode}
s~=~async'~s0~(map~f~s)
\end{lyxcode}
\item Add monadic signal combinator, \texttt{bind} \texttt{:} $(\alpha\rightarrow\Sigma\beta)\rightarrow\Sigma\alpha\rightarrow\Sigma\beta$

\begin{itemize}
\item use input signals from dynamically created UI:\end{itemize}
\begin{lyxcode}
viewS~=~map~draw~stateS

stateS~=~foldp~update\_on\_click~(bind~get\_clicks~viewS)
\end{lyxcode}
\item Allow user-defined signals constructed from asynchronous APIs

\begin{itemize}
\item Generate signal updates whenever callback is called:\end{itemize}
\begin{lyxcode}
type~$\mathbf{C}\alpha\beta=\alpha\rightarrow\left(\beta\rightarrow\bot\right)\rightarrow\bot$

chain~:~$\mathbf{C}\alpha\beta\rightarrow\Sigma\alpha\rightarrow\Sigma\beta$~

some\_async\_api~:~$\mathbf{C}\alpha\beta$

values\_of\_b~=~chain~some\_async\_api~values\_of\_a~
\end{lyxcode}
\end{itemize}

\lyxframeend{}


\lyxframeend{}\lyxframe{Part 2. Temporal logic and FRP}
\begin{itemize}
\item Reminder (Curry-Howard): logical expressions will be types

\begin{itemize}
\item ...and the axioms will be primitive terms
\end{itemize}
\item We only need to control the \textbf{order} of events: no ``hard real-time''
\item How to understand temporal logic:

\begin{itemize}
\item classical propositional logic $\approx$ Boolean arithmetic
\item intuitionistic propositional logic $\approx$ same but without \textbf{true}
/ \textbf{false} dichotomy
\item (linear-time) temporal logic LTL$\approx$ Boolean arithmetic for
\emph{infinite sequences}
\item intuitionistic temporal logic ITL$\approx$ same but without \textbf{true}
/ \textbf{false} dichotomy
\end{itemize}
\item In other words:

\begin{itemize}
\item an ITL type represents a \textbf{single infinite sequence} of values
\end{itemize}
\end{itemize}

\lyxframeend{}


\lyxframeend{}\lyxframe{Boolean arithmetic: notation}
\begin{itemize}
\item Classical propositional (Boolean) logic: $T$, $F$, $a\vee b$, $a\wedge b$,
$\neg a$, $a\rightarrow b$
\item A notation better adapted to school-level arithmetic: $1$, $0$,
$a+b$, $ab$, $a'$
\item The only ``new rule'' is $1+1=1$
\item Define $a\rightarrow b=a'+b$
\item Some identities: 
\begin{align*}
0a=0,\quad1a=a, & \quad a+0=a,\quad a+1=1,\\
a+a=a,\quad aa=a, & \quad a+a'=1,\quad aa'=0,\\
\left(a+b\right)'=a'b', & \quad\left(ab\right)'=a'+b',\quad\left(a'\right)'=a\\
a\left(b+c\right)=ab+ac, & \quad\left(a+b\right)\left(a+c\right)=a+bc
\end{align*}
 
\end{itemize}

\lyxframeend{}


\lyxframeend{}\lyxframe{Boolean arithmetic: example}

\textcolor{blue}{\emph{Of the three suspects $A$, $B$, $C$, only
one is guilty of a crime. }}

\textcolor{blue}{\emph{Suspect $A$ says: ``$B$ did it''. Suspect
$B$ says: ``$C$ is innocent.''}}

\textcolor{blue}{\emph{The guilty one is lying, the innocent ones
tell the truth.}}
\[
\phi=\left(ab'c'+a'bc'+a'b'c\right)\left(a'b+ab'\right)\left(b'c'+bc\right)
\]
\textbf{Simplify}: expand the brackets, omit $aa'$, $bb'$, $cc'$,
replace $aa=a$ etc.:
\[
\phi=ab'c'+0+0=ab'c'
\]


\textcolor{blue}{The guilty one is $A$.}


\lyxframeend{}


\lyxframeend{}\lyxframe{Propositional linear-time temporal logic (LTL)}
\begin{itemize}
\item We work with\emph{ infinite boolean sequences} (``linear time'')\\
\textbf{Boolean} operations:
\begin{align*}
a & =\left[a_{0},a_{1},a_{2},...\right];\quad b=\left[b_{0},b_{1},b_{2},...\right];\\
a+b & =\left[a_{0}+b_{0},a_{1}+b_{1},...\right];\; a'=\left[a_{0}^{\prime},a_{1}^{\prime},...\right];\; ab=\left[a_{0}b_{0},a_{1}b_{1},...\right]
\end{align*}
\textbf{Temporal} operations:
\begin{align*}
\mbox{(Next)}\quad\mathbf{N}a & =\left[a_{1},a_{2},...\right]\\
\mbox{(Sometimes)}\quad\mathbf{F}a & =\left[a_{0}+a_{1}+a_{2}+...,\ a_{1}+a_{2}+...,\ ...\right]\\
\mbox{(Always)}\quad\mathbf{G}a & =\left[a_{0}a_{1}a_{2}a_{3}...,\ a_{1}a_{2}a_{3}...,\ a_{2}a_{3}...,\ ...\right]
\end{align*}
Other notation (from modal logic):
\[
\mathbf{N}a\equiv\bigcirc a;\;\mathbf{F}a\equiv\lozenge a;\;\mathbf{G}a\equiv\square a
\]

\item Weak Until: $p\mathbf{U}q$ = ``$p$ holds from now on until $q$
first becomes true''
\[
p\mathbf{U}q=q+p\mathbf{N}\left(q+p\mathbf{N}\left(q+...\right)\right)
\]

\end{itemize}

\lyxframeend{}


\lyxframeend{}\lyxframe{Temporal logic redux}

Designers of FRP languages must face some choices:
\begin{itemize}
\item LTL as type theory: do we use $\mathbf{N}\alpha$, $\mathbf{F}\alpha$,
$\mathbf{G}\alpha$ as new types?
\item Are they to be functors, monads, ...?
\item Which temporal axioms to use as language primitives?
\item What is the operational semantics? (I.e., how to compile this?)
\end{itemize}
A sophisticated example: {[}Krishnaswamy 2013{]}
\begin{itemize}
\item uses full LTL with higher-order temporal types and fixpoints
\item uses linear types to control space/time leaks
\end{itemize}

\lyxframeend{}


\lyxframeend{}\lyxframe{Interpreting values typed by LTL}
\begin{itemize}
\item What does it mean to have a value $x$ of type, say, $\mathbf{G}(\alpha\rightarrow\alpha\mathbf{U}\beta)$
??

\begin{itemize}
\item $x:\mathbf{N}\alpha$ means that $x:\alpha$ will be available \emph{only}
at the \emph{next} time tick \\
($x$ is a \textbf{deferred value} of type $\alpha$)
\item $x:\mathbf{F}\alpha$ means that $x:\alpha$ will be available at
\emph{some} future tick(s)\\
($x$ is an \textbf{event} of type $\alpha$)
\item $x:\mathbf{G\alpha}$ means that a (different) value $x:\alpha$ is
available at \emph{every} tick\\
($x$ is an \textbf{infinite stream} of type $\alpha$)
\item $x:\alpha\mathbf{U}\beta$ means a \textbf{finite stream} of $\alpha$
that may end with a $\beta$ 
\end{itemize}
\item Some temporal \textbf{axioms} of intuitionistic LTL:
\begin{align*}
\mbox{(deferred apply)}\quad\mathbf{N}\left(\alpha\rightarrow\beta\right) & \rightarrow\left(\mathbf{N}\alpha\rightarrow\mathbf{N}\beta\right);\\
\mathbf{\mbox{(streamed apply)}\quad G}\left(\alpha\rightarrow\beta\right) & \rightarrow\left(\mathbf{G}\alpha\rightarrow\mathbf{G}\beta\right);\\
\mbox{(generate a stream)}\quad\mathbf{G}\left(\alpha\rightarrow\mathbf{N}\alpha\right) & \rightarrow\left(\alpha\rightarrow\mathbf{G}\alpha\right);\\
\mbox{(read infinite stream)}\quad\mathbf{G}\alpha & \rightarrow\alpha\mathbf{N}(\mathbf{G}\alpha)\\
\mbox{(read finite stream)}\quad\alpha\mathbf{U}\beta & \rightarrow\beta+\alpha\mathbf{N}(\alpha\mathbf{U}\beta)
\end{align*}

\end{itemize}

\lyxframeend{}


\lyxframeend{}\lyxframe{\texttt{Elm} as an FRP language }
\begin{itemize}
\item $\lambda$-calculus with type $\mathbf{G}\alpha$, primitives \texttt{map2},
\texttt{foldp}, \texttt{async} \end{itemize}
\begin{lyxcode}
map2~:~$\left(\alpha\rightarrow\beta\rightarrow\gamma\right)\rightarrow\mathbf{G}\alpha\rightarrow\mathbf{G}\beta\rightarrow\mathbf{G}\gamma$

foldp~:~$\left(\alpha\rightarrow\beta\rightarrow\beta\right)\rightarrow\beta\rightarrow\mathbf{G}\alpha\rightarrow\mathbf{G}\beta$

async~:~$\mathbf{G}\alpha\rightarrow\mathbf{G}\alpha$\end{lyxcode}
\begin{itemize}
\item (\texttt{map2} makes $\mathbf{G}$ an applicative functor)
\item \texttt{async} is a special \emph{scheduling} \emph{instruction}
\item Limitations:

\begin{itemize}
\item Cannot have a type $\mathbf{G}(\mathbf{G}\alpha)$, also not using
$\mathbf{N}$ or $\mathbf{F}$
\item Cannot construct temporal values by hand
\item This language is an \emph{incomplete} Curry-Howard image of LTL!
\end{itemize}
\end{itemize}

\lyxframeend{}


\lyxframeend{}\lyxframe{Conclusions}
\begin{itemize}
\item There are some languages that implement FRP in various \emph{ad hoc}
ways
\item The ideal is not (yet) reached
\item \texttt{Elm}-style FRP is a promising step in the right direction
\end{itemize}

\lyxframeend{}


\lyxframeend{}\lyxframe{Abstract}

In my day job, most bugs come from implementing reactive programs
imperatively. FRP is a declarative approach that promises to solve
these problems. \medskip{}


FRP can be defined as a $\lambda$-calculus that admits temporal types,
i.e.\ types given by a propositional intuitionistic linear-time temporal
logic (LTL). Although the \texttt{Elm} language uses only a subset
of LTL, it achieves high expressivity for GUI programming. I will
formally define the operational semantics of \texttt{Elm}. I discuss
the current limitations of \texttt{Elm} and outline possible extensions.
I also review the connections between temporal logic, FRP, and \texttt{Elm}.
\medskip{}


My talk will be understandable to anyone familiar with Curry-Howard
and functional programming. The first part of the talk is a self-contained
presentation of \texttt{Elm} that does not rely on temporal logic
or Curry-Howard. The second part of the talk will explain the basic
intuitions behind temporal logic and its connection with FRP.


\lyxframeend{}


\lyxframeend{}\lyxframe{Suggested reading }

E. Czaplicki, S. Chong. \href{http://people.seas.harvard.edu/~chong/pubs/pldi13-elm.pdf}{Asynchronous FRP for GUIs}.
(2013) 

E. Czaplicki. \href{http://www.seas.harvard.edu/sites/default/files/files/archived/Czaplicki.pdf}{Concurrent FRP for functional GUI}
(2012). 

N. R. Krishnaswamy. \href{https://www.mpi-sws.org/~neelk/simple-frp.pdfHigher-order functional reactive programming without spacetime leaks}{https://www.mpi-sws.org/$\sim$neelk/simple-frp.pdfHigher-order functional reactive programming without spacetime leaks}(2013).

M. F. Dam. Lectures on temporal logic. Slides: \href{http://www.csc.kth.se/~mfd/Courses/Temporal_logic/lecture1.pdf}{Syntax and semantics of LTL},
\href{http://www.csc.kth.se/~mfd/Courses/Temporal_logic/lecture2.pdf}{A Hilbert-style proof system for LTL} 

E. Bainomugisha, et al. \href{ftp://progftp.vub.ac.be/tech_report/2012/vub-soft-tr-12-13.pdf}{A survey of reactive programming}
(2013).

W. Jeltsch. \href{http://www.ioc.ee/~wolfgang/research/plpv-2013-paper.pdf}{Temporal logic with Until, Functional Reactive Programming with processes, and concrete process categories.}
(2013).

A. Jeffrey. \href{http://ect.bell-labs.com/who/ajeffrey/papers/plpv12.pdf}{LTL types FRP.}
(2012).

D. Marchignoli. \href{http://phd.di.unipi.it/Theses/PhDthesis_Marchignoli.pdf}{Natural deduction systems for temporal logic.}
(2002). -- See Chapter 2 for a natural deduction system for modal
and temporal logics. 


\lyxframeend{}
\end{document}
