%% LyX 2.0.6 created this file.  For more info, see http://www.lyx.org/.
%% Do not edit unless you really know what you are doing.
\documentclass[english,hyperref]{beamer}
\usepackage[T1]{fontenc}
\usepackage[latin9]{inputenc}
\setcounter{secnumdepth}{3}
\setcounter{tocdepth}{3}
\usepackage{babel}
\usepackage{amsmath}
\usepackage{amssymb}
\ifx\hypersetup\undefined
  \AtBeginDocument{%
    \hypersetup{unicode=true,pdfusetitle,
 bookmarks=true,bookmarksnumbered=true,bookmarksopen=false,
 breaklinks=false,pdfborder={0 0 1},backref=false,colorlinks=true,
 linkcolor=black,urlcolor=blue}
  }
\else
  \hypersetup{unicode=true,pdfusetitle,
 bookmarks=true,bookmarksnumbered=true,bookmarksopen=false,
 breaklinks=false,pdfborder={0 0 1},backref=false,colorlinks=true,
 linkcolor=black,urlcolor=blue}
\fi

\makeatletter

%%%%%%%%%%%%%%%%%%%%%%%%%%%%%% LyX specific LaTeX commands.
%% Because html converters don't know tabularnewline
\providecommand{\tabularnewline}{\\}

%%%%%%%%%%%%%%%%%%%%%%%%%%%%%% Textclass specific LaTeX commands.
 % this default might be overridden by plain title style
 \newcommand\makebeamertitle{\frame{\maketitle}}%
 \AtBeginDocument{
   \let\origtableofcontents=\tableofcontents
   \def\tableofcontents{\@ifnextchar[{\origtableofcontents}{\gobbletableofcontents}}
   \def\gobbletableofcontents#1{\origtableofcontents}
 }
 \long\def\lyxframe#1{\@lyxframe#1\@lyxframestop}%
 \def\@lyxframe{\@ifnextchar<{\@@lyxframe}{\@@lyxframe<*>}}%
 \def\@@lyxframe<#1>{\@ifnextchar[{\@@@lyxframe<#1>}{\@@@lyxframe<#1>[]}}
 \def\@@@lyxframe<#1>[{\@ifnextchar<{\@@@@@lyxframe<#1>[}{\@@@@lyxframe<#1>[<*>][}}
 \def\@@@@@lyxframe<#1>[#2]{\@ifnextchar[{\@@@@lyxframe<#1>[#2]}{\@@@@lyxframe<#1>[#2][]}}
 \long\def\@@@@lyxframe<#1>[#2][#3]#4\@lyxframestop#5\lyxframeend{%
   \frame<#1>[#2][#3]{\frametitle{#4}#5}}
 \def\lyxframeend{} % In case there is a superfluous frame end
 \newenvironment{lyxcode}
   {\par\begin{list}{}{
     \setlength{\rightmargin}{\leftmargin}
     \setlength{\listparindent}{0pt}% needed for AMS classes
     \raggedright
     \setlength{\itemsep}{0pt}
     \setlength{\parsep}{0pt}
     \normalfont\ttfamily}%
    \def\{{\char`\{}
    \def\}{\char`\}}
    \def\textasciitilde{\char`\~}
    \item[]}
   {\end{list}}

%%%%%%%%%%%%%%%%%%%%%%%%%%%%%% User specified LaTeX commands.
\usetheme[secheader]{Boadilla}
\usecolortheme{seahorse}
\title[Elm-style FRP]{Elm-style Functional Reactive Programming demystified}
\author{Sergei Winitzki}
\date{April 13, 2015}
\institute[Versal Group Inc.]{SF Types, Theorems, and Programming Languages}

\makeatother

\begin{document}
\frame{\titlepage}


\lyxframeend{}\lyxframe{What is ``functional reactive programming''}

FRP has little to do with...
\begin{itemize}
\item multithreading, message-passing concurrency, ``actors''
\item distributed computing on massively parallel load-balanced clusters
\item ma/reduce, the ``reactive manifesto'', (\emph{insert latest fad
here})...
\end{itemize}
FRP is...
\begin{itemize}
\item \textbf{pure functions using temporal types as primitives}

\begin{itemize}
\item (temporal type $\approx$ lazy stream of events)
\end{itemize}
\end{itemize}

\lyxframeend{}


\lyxframeend{}\lyxframe{Transformational vs. reactive programs}

\begin{center}
\begin{tabular}{|c|c|}
\hline 
\textbf{Transformational} programs & \textbf{Reactive} programs\tabularnewline
\hline 
\hline 
\textbf{example}: \texttt{\textcolor{blue}{\footnotesize{pdflatex
elm\_talk.tex}}} & \textbf{example}: any GUI program, OS\tabularnewline
\hline 
start, run, then stop & keep running indefinitely\tabularnewline
\hline 
read some input, write some output & wait for signals, send messages \tabularnewline
\hline 
\textbf{execution:} sequential, parallel & ``main run loop'' + concurrency\tabularnewline
\hline 
\textbf{difficulty:} algorithms & signal/response sequences\tabularnewline
\hline 
\textbf{specification:} classical logic? & classical temporal logic?\tabularnewline
\hline 
\textbf{verification:} proof of correctness? & model checking?\tabularnewline
\hline 
\textbf{synthesis:} extract code from proof? & temporal logic synthesis? \tabularnewline
\hline 
\textbf{type theory:} intuitionistic logic & intuitionistic \emph{temporal} logic\tabularnewline
\hline 
\end{tabular}
\par\end{center}


\lyxframeend{}


\lyxframeend{}\lyxframe{Difficulties in reactive programming}
\begin{itemize}
\item Input signals may come at unpredictable times

\begin{itemize}
\item Imperative updates are difficult to keep in the correct order
\item Flow of events becomes difficult to understand
\end{itemize}
\item Asynchronous (out-of-order) callback logic becomes opaque
\item Inverted control (``the system will call you'') obscures the flow
of data
\item Some concurrency is usually required (e.g.~background tasks)

\begin{itemize}
\item Explicit multithreaded code is hard to write and debug
\end{itemize}
\end{itemize}

\lyxframeend{}


\lyxframeend{}\lyxframe{Motivation for FRP}
\begin{itemize}
\item Reactive programs work on \textbf{infinite sequences} of input/output
values
\item Main idea: make infinite sequences implicit, as a new ``temporal''
type

\begin{itemize}
\item (Elm) \texttt{Signal} $\alpha$ --- an infinite sequence of values
of type $\alpha$
\item alternatively, a value of type $\alpha$ that ``changes with time''
\end{itemize}
\item Reactive programs are \textbf{pure functions}

\begin{itemize}
\item a GUI is a pure function of type \texttt{Signal} \texttt{Inputs} $\rightarrow$
\texttt{Signal} \texttt{View} 
\item a Web server is a pure function \texttt{Signal} \texttt{Request} $\rightarrow$
\texttt{Signal} \texttt{Response} 
\item all mutation is \textbf{implicit} in \texttt{Signal} $\alpha$; our
code is 100\% immutable

\begin{itemize}
\item instead of updating an \texttt{x:Int}, we define a value of type \texttt{Signal}
\texttt{Int}
\end{itemize}
\item asynchronous behavior is \textbf{implicit}: our code has no callbacks
\item concurrency / parallelism is \textbf{implicit}

\begin{itemize}
\item the runtime needs to provide the required scheduling of events
\end{itemize}
\end{itemize}
\end{itemize}

\lyxframeend{}


\lyxframeend{}\lyxframe{\texttt{Elm} in a nutshell}
\begin{itemize}
\item \texttt{Elm} is a pure polymorphic $\lambda$-calculus with products
and sums
\item \textbf{Temporal type} $\Sigma\alpha$ --- a time-dependent value
of \textbf{ordinary} type $\alpha$
\item Temporal primitive terms in core \texttt{Elm}: \end{itemize}
\begin{lyxcode}
constant:~$\alpha\rightarrow\Sigma\alpha$

map2:~$(\alpha\rightarrow\beta\rightarrow\gamma)\rightarrow\Sigma\alpha\rightarrow\Sigma\beta\rightarrow\Sigma\gamma$

foldp:~\textrm{$\left(\alpha\rightarrow\beta\rightarrow\beta\right)\rightarrow\beta\rightarrow\Sigma\alpha\rightarrow\Sigma\beta$}

async:~$\Sigma\alpha\rightarrow\Sigma\alpha$\end{lyxcode}
\begin{itemize}
\item \textbf{No nested} temporal types: \texttt{constant (constant x)}
is ill-typed!
\item Domain-specific primitive types: \texttt{Bool}, \texttt{Int}, \texttt{Float},
\texttt{String}, \texttt{View}
\item Standard library with data structures, HTML, HTTP, JSON, ...

\begin{itemize}
\item ...and signals \texttt{Time.every}, \texttt{Mouse.position}, \texttt{Window.dimensions},
...
\end{itemize}
\item Try \texttt{Elm} online at \href{http://elm-lang.org/try}{http://elm-lang.org/try}
\end{itemize}

\lyxframeend{}


\lyxframeend{}\lyxframe{\texttt{Elm} type judgments}
\begin{itemize}
\item Non-temporal values are used as in polymorphic $\lambda$-calculus
\end{itemize}
\begin{align*}
\frac{\Gamma,(x:\alpha)\vdash e:\beta}{\Gamma\vdash(\lambda x.e):\alpha\rightarrow\beta}\,\textsc{Lambda} & \quad\frac{\Gamma\vdash e_{1}:\alpha\rightarrow\beta\quad\Gamma\vdash e_{2}:\alpha}{\Gamma\vdash e_{1}\, e_{2}:\beta}\,\textsc{Apply}
\end{align*}

\begin{itemize}
\item Temporal values have special types of the form $\Sigma\tau$

\begin{itemize}
\item Here, type variables $\alpha,\beta$,... \textbf{cannot} \textbf{be}
of the form $\Sigma\tau$:
\end{itemize}
\end{itemize}
\begin{align*}
\frac{\Gamma\vdash e:\alpha}{\Gamma\vdash(\mbox{constant}\, e):\Sigma\alpha}\,\textsc{Constant}\\
\frac{\Gamma\vdash m:\alpha\rightarrow\beta\rightarrow\gamma\quad\Gamma\vdash p:\Sigma\alpha\quad\Gamma\vdash q:\Sigma\beta}{\Gamma\vdash\mbox{map2}\: m\: p\: q:\Sigma\gamma}\,\textsc{Map2}\\
\frac{\Gamma\vdash u:\alpha\rightarrow\beta\rightarrow\beta\quad\Gamma\vdash e:\beta\quad\Gamma\vdash s:\Sigma\alpha}{\Gamma\vdash(\mbox{foldp}\, u\, e\, s):\Sigma\beta}\,\textsc{FoldP}
\end{align*}

\begin{itemize}
\item A value of type $\Sigma\Sigma\alpha$ is impossible in a well-typed
expression!
\end{itemize}

\lyxframeend{}


\lyxframeend{}\lyxframe{\texttt{{*}Elm} operational semantics}
\begin{itemize}
\item Non-temporal expressions are evaluated \textbf{eagerly} in pure $\lambda$-calculus
\item The runtime will cache all values to avoid recomputation 
\end{itemize}

\lyxframeend{}


\lyxframeend{}\lyxframe{\texttt{{*}Elm} operational semantics}
\begin{itemize}
\item Example
\item \textcolor{blue}{\emph{I work after the boss comes by and until the
phone rings}}: \\
$\quad$\texttt{let after\_until w (b,r) = (w or b) and not r in }\\
$\quad\quad$\texttt{foldp after\_until false (boss, phone)}
\end{itemize}

\lyxframeend{}


\lyxframeend{}\lyxframe{GUI building: ``Hello, world'' in \texttt{Elm} }
\begin{itemize}
\item The value called \texttt{\textbf{main}} will be visualized by the
runtime\end{itemize}
\begin{lyxcode}
import~Graphics.Element~(..)~

import~Text~(..)~

import~Signal~(..)

~~

text~:~Element

text~=~plainText~\textquotedbl{}Hello,~World!\textquotedbl{}

~~

main~:~Signal~Element~

main~=~constant~text
\end{lyxcode}

\lyxframeend{}


\lyxframeend{}\lyxframe{Typical program structure in \texttt{Elm} }
\begin{itemize}
\item A state machine:\end{itemize}
\begin{lyxcode}
update:~Command~$\rightarrow$~State~$\rightarrow$~State\end{lyxcode}
\begin{itemize}
\item A rendering function:\end{itemize}
\begin{lyxcode}
draw:~State~$\rightarrow$~View\end{lyxcode}
\begin{itemize}
\item A manager that merges the required input signals into one:

\begin{itemize}
\item may use \texttt{Mouse}, \texttt{Keyboard}, \texttt{Time}, \texttt{HTML}
stuff, etc.
\end{itemize}
\end{itemize}
\begin{lyxcode}
merge\_inputs:~Signal~Command\end{lyxcode}
\begin{itemize}
\item Program boilerplate:\end{itemize}
\begin{lyxcode}
init\_state~:~State

main~:~Signal~View

main~=~map~draw~\$~foldp~update~init\_state~merge\_inputs~
\end{lyxcode}

\lyxframeend{}


\lyxframeend{}\lyxframe{{*}Asynchrony and concurrency in \texttt{Elm}}
\begin{itemize}
\item Long-running computations will delay signal updates
\item Solutions: 1. Caching of all results. 2. Using \texttt{async} 
\end{itemize}

\lyxframeend{}


\lyxframeend{}\lyxframe{Some limitations of \texttt{Elm}-style FRP}
\begin{itemize}
\item No recursion of any kind
\item No higher-order signals: $\Sigma(\Sigma\alpha)$ is disallowed by
the type system
\item No distinction between continuous time and discrete time
\item The signal processing logic is fully specified statically
\item No constructors for signals

\begin{itemize}
\item Impossible to implement the ``dining philosophers''!
\end{itemize}
\end{itemize}

\lyxframeend{}


\lyxframeend{}\lyxframe{\texttt{Elm} cannot do ``dining philosophers''}
\begin{itemize}
\item ``Dining philosophers'': need to simulate a philosopher who thinks
for a random time and then eats for a random time
\item Can a signal value \texttt{p} \texttt{:} \texttt{Signal} \texttt{Unit}
update at random times? 

\begin{itemize}
\item No! There is no way to delay the update times of a signal \textbf{at
runtime}.
\item \texttt{Time.delay:} \texttt{Int}$\rightarrow\Sigma\alpha\rightarrow\Sigma\alpha$
cannot use a time-varying delay value 
\item \texttt{Time.every:} \texttt{Int}$\rightarrow\Sigma$\texttt{Int}
also requires a fixed delay value
\item Cannot lift \texttt{Time.every} into \texttt{$\Sigma$Int}$\rightarrow\Sigma\Sigma$\texttt{Int}
to achieve variable delay
\end{itemize}
\end{itemize}

\lyxframeend{}


\lyxframeend{}\lyxframe{The JavaScript backend for \texttt{Elm}}

Features:
\begin{itemize}
\item Good support for HTML/CSS, HTTP requests, JSON
\item Good performance of caching views
\item Transparent, declarative modeling of data
\end{itemize}
Limitations:
\begin{itemize}
\item No implementation for \texttt{async} and no concurrency
\item Ordinary recursion may generate invalid JavaScript
\item The lack of recursive signals is compensated by \emph{ad hoc} primitives
\end{itemize}

\lyxframeend{}


\lyxframeend{}\lyxframe{{*}Possible extensions}
\begin{itemize}
\item Recursive definitions for signals
\item Monadic signal combinators
\item Signal constructors
\end{itemize}

\lyxframeend{}


\lyxframeend{}\lyxframe{Part 2. Temporal logic and FRP}

{\footnotesize{This part of the talk is optional.}}{\footnotesize \par}
\begin{itemize}
\item Reminder (Curry-Howard): temporal logic expressions will be our types
\item We only need to control the \textbf{order} of events: no ``hard real-time''
\item How to understand temporal logic:

\begin{itemize}
\item classical propositional logic $\approx$ Boolean arithmetic
\item intuitionistic propositional logic $\approx$ same but without \textbf{true}
/ \textbf{false} dichotomy
\item (linear-time) temporal logic $\approx$ Boolean arithmetic for \emph{infinite
sequences}
\item intuitionistic temporal logic $\approx$ same but without \textbf{true}
/ \textbf{false} dichotomy
\end{itemize}
\item In other words:

\begin{itemize}
\item a temporal type represents a \textbf{single infinite sequence} of
values
\end{itemize}
\end{itemize}

\lyxframeend{}


\lyxframeend{}\lyxframe{Boolean arithmetic: notation}
\begin{itemize}
\item Classical propositional (Boolean) logic: $T$, $F$, $a\vee b$, $a\wedge b$,
$\neg a$, $a\rightarrow b$
\item A notation better adapted to school-level arithmetic: $1$, $0$,
$a+b$, $ab$, $a'$
\item The only ``new rule'' is $1+1=1$
\item Define $a\rightarrow b=a'+b$
\item Some identities: 
\begin{align*}
0a=0,\quad1a=a, & \quad a+0=a,\quad a+1=1,\\
a+a=a,\quad aa=a, & \quad a+a'=1,\quad aa'=0,\\
\left(a+b\right)'=a'b', & \quad\left(ab\right)'=a'+b',\quad\left(a'\right)'=a\\
a\left(b+c\right)=ab+ac, & \quad\left(a+b\right)\left(a+c\right)=a+bc
\end{align*}
 
\end{itemize}

\lyxframeend{}


\lyxframeend{}\lyxframe{Boolean arithmetic: example}

\textcolor{blue}{\emph{Of the three suspects $A$, $B$, $C$, only
one is guilty of a crime. }}

\textcolor{blue}{\emph{Suspect $A$ says: ``$B$ did it''. Suspect
$B$ says: ``$C$ is innocent.''}}

\textcolor{blue}{\emph{The guilty one is lying, the innocent ones
tell the truth.}}
\[
\phi=\left(ab'c'+a'bc'+a'b'c\right)\left(a'b+ab'\right)\left(b'c'+bc\right)
\]
\textbf{Simplify}: expand the brackets, omit $aa'$, $bb'$, $cc'$,
replace $aa=a$ etc.:
\[
\phi=ab'c'+0+0=ab'c'
\]


\textcolor{blue}{The guilty one is $A$.}


\lyxframeend{}


\lyxframeend{}\lyxframe{Propositional linear-time temporal logic (LTL)}
\begin{itemize}
\item We work with\emph{ infinite boolean sequences} (``linear time'')\\
\textbf{Boolean} operations:
\begin{align*}
a & =\left[a_{0},a_{1},a_{2},...\right];\quad b=\left[b_{0},b_{1},b_{2},...\right];\\
a+b & =\left[a_{0}+b_{0},a_{1}+b_{1},...\right];\; a'=\left[a_{0}^{\prime},a_{1}^{\prime},...\right];\; ab=\left[a_{0}b_{0},a_{1}b_{1},...\right]
\end{align*}
\textbf{Temporal} operations:
\begin{align*}
\mbox{(Next)}\quad\mathbf{N}a & =\left[a_{1},a_{2},...\right]\\
\mbox{(Sometimes)}\quad\mathbf{F}a & =\left[a_{0}+a_{1}+a_{2}+...,\ a_{1}+a_{2}+...,\ ...\right]\\
\mbox{(Always)}\quad\mathbf{G}a & =\left[a_{0}a_{1}a_{2}a_{3}...,\ a_{1}a_{2}a_{3}...,\ a_{2}a_{3}...,\ ...\right]
\end{align*}
Other notation (from modal logic):
\[
\mathbf{N}a\equiv\bigcirc a;\;\mathbf{F}a\equiv\lozenge a;\;\mathbf{G}a\equiv\square a
\]

\item Weak Until: $p\mathbf{U}q$ = ``$p$ holds from now on until $q$
first becomes true''
\[
p\mathbf{U}q=q+p\mathbf{N}\left(q+p\mathbf{N}\left(q+...\right)\right)
\]

\end{itemize}

\lyxframeend{}


\lyxframeend{}\lyxframe{Temporal logic redux}
\begin{itemize}
\item LTL as type theory: do we use $\mathbf{N}\alpha$, $\mathbf{F}\alpha$,
$\mathbf{G}\alpha$ as new types?
\item Are they to be functors, monads, ...?
\item What is the operational semantics? (I.e., how to compile this?)
\end{itemize}

\lyxframeend{}


\lyxframeend{}\lyxframe{Interpreting values typed by LTL}
\begin{itemize}
\item What does it mean to have a value $x$ of type, say, $\mathbf{G}(\alpha\rightarrow\alpha\mathbf{U}\beta)$
??

\begin{itemize}
\item $x:\mathbf{N}\alpha$ means that $x:\alpha$ will be available \emph{only}
at the \emph{next} time tick \\
($x$ is a \textbf{deferred value} of type $\alpha$)
\item $x:\mathbf{F}\alpha$ means that $x:\alpha$ will be available at
\emph{some} future tick(s)\\
($x$ is an \textbf{event} of type $\alpha$)
\item $x:\mathbf{G\alpha}$ means that a (different) value $x:\alpha$ is
available at \emph{every} tick\\
($x$ is an \textbf{infinite stream} of type $\alpha$)
\item $x:\alpha\mathbf{U}\beta$ means a \textbf{finite stream} of $\alpha$
that may end with a $\beta$ 
\end{itemize}
\item Some \emph{temporal axioms} of intuitionistic LTL:
\begin{align*}
\mbox{(deferred apply)}\quad\mathbf{N}\left(\alpha\rightarrow\beta\right) & \rightarrow\left(\mathbf{N}\alpha\rightarrow\mathbf{N}\beta\right);\\
\mathbf{\mbox{(streamed apply)}\quad G}\left(\alpha\rightarrow\beta\right) & \rightarrow\left(\mathbf{G}\alpha\rightarrow\mathbf{G}\beta\right);\\
\mbox{(generate a stream)}\quad\mathbf{G}\left(\alpha\rightarrow\mathbf{N}\alpha\right) & \rightarrow\left(\alpha\rightarrow\mathbf{G}\alpha\right);\\
\mbox{(read infinite stream)}\quad\mathbf{G}\alpha & \rightarrow\alpha\mathbf{N}(\mathbf{G}\alpha)\\
\mbox{(read finite stream)}\quad\alpha\mathbf{U}\beta & \rightarrow\beta+\alpha\mathbf{N}(\alpha\mathbf{U}\beta)
\end{align*}

\end{itemize}

\lyxframeend{}


\lyxframeend{}\lyxframe{\texttt{Elm} as an FRP language }
\begin{itemize}
\item $\lambda$-calculus with type $\mathbf{G}\alpha$, primitives \texttt{map2},
\texttt{foldp}, \texttt{async} \end{itemize}
\begin{lyxcode}
map2~:~$\left(\alpha\rightarrow\beta\rightarrow\gamma\right)\rightarrow\mathbf{G}\alpha\rightarrow\mathbf{G}\beta\rightarrow\mathbf{G}\gamma$

foldp~:~$\left(\alpha\rightarrow\beta\rightarrow\beta\right)\rightarrow\beta\rightarrow\mathbf{G}\alpha\rightarrow\mathbf{G}\beta$

async~:~$\mathbf{G}\alpha\rightarrow\mathbf{G}\alpha$\end{lyxcode}
\begin{itemize}
\item (\texttt{map2} makes $\mathbf{G}$ an applicative functor)
\item \texttt{async} is a special \emph{scheduling} \emph{instruction}
\item Limitations:

\begin{itemize}
\item Cannot have a type $\mathbf{G}(\mathbf{G}\alpha)$, also not using
$\mathbf{N}$ or $\mathbf{F}$
\item Cannot construct temporal values by hand
\item This language is an \emph{incomplete} Curry-Howard image of LTL!
\end{itemize}
\end{itemize}

\lyxframeend{}


\lyxframeend{}\lyxframe{Conclusions and outlook}
\begin{itemize}
\item There are some languages that implement FRP in various \emph{ad hoc}
ways
\item The ideal is not (yet) reached
\end{itemize}

\lyxframeend{}


\lyxframeend{}\lyxframe{{*}Conclusions and outlook}
\begin{itemize}
\item The ideal is not (yet) reached
\end{itemize}

\lyxframeend{}


\lyxframeend{}\lyxframe{Abstract}

In my day job, most bugs come from imperatively implemented reactive
programs. FRP is a declarative approach that promises to solve my
problems. \medskip{}


FRP can be defined as a $\lambda$-calculus with types given by a
propositional intuitionistic linear-time temporal logic (LTL). Although
the \texttt{Elm} language uses only a subset of LTL, it achieves high
expressivity for GUI programming. I discuss the current limitations
of \texttt{Elm} and outline some possible extensions. I will also
briefly review the motivations behind and the connections between
temporal logic, FRP, and \texttt{Elm}. \medskip{}


My talk will be understandable to anyone familiar with Curry-Howard
and functional programming. (The first part of the talk does not rely
on temporal logic or Curry-Howard.)


\lyxframeend{}


\lyxframeend{}\lyxframe{Suggested reading }

E. Czaplicki, S. Chong. \href{http://people.seas.harvard.edu/~chong/pubs/pldi13-elm.pdf}{Asynchronous FRP for GUIs}.
(2013) 

E. Czaplicki. \href{http://www.seas.harvard.edu/sites/default/files/files/archived/Czaplicki.pdf}{Concurrent FRP for functional GUI}
(2012). 

M. F. Dam. Lectures on temporal logic. Slides: \href{http://www.csc.kth.se/~mfd/Courses/Temporal_logic/lecture1.pdf}{Syntax and semantics of LTL},
\href{http://www.csc.kth.se/~mfd/Courses/Temporal_logic/lecture2.pdf}{A Hilbert-style proof system for LTL} 

E. Bainomugisha, et al. \href{ftp://progftp.vub.ac.be/tech_report/2012/vub-soft-tr-12-13.pdf}{A survey of reactive programming}
(2013).

W. Jeltsch. \href{http://www.ioc.ee/~wolfgang/research/plpv-2013-paper.pdf}{Temporal logic with Until, Functional Reactive Programming with processes, and concrete process categories.}
(2013).

A. Jeffrey. \href{http://ect.bell-labs.com/who/ajeffrey/papers/plpv12.pdf}{LTL types FRP.}
(2012).

D. Marchignoli. \href{http://phd.di.unipi.it/Theses/PhDthesis_Marchignoli.pdf}{Natural deduction systems for temporal logic.}
(2002). -- See Chapter 2 for a natural deduction system for modal
and temporal logics. 


\lyxframeend{}
\end{document}
