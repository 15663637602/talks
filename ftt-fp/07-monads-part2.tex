\batchmode
\makeatletter
\def\input@path{{/Users/sergei.winitzki/Code/talks/ftt-fp/}}
\makeatother
\documentclass[english]{beamer}
\usepackage[T1]{fontenc}
\usepackage[latin9]{inputenc}
\setcounter{secnumdepth}{3}
\setcounter{tocdepth}{3}
\usepackage{babel}
\usepackage{amsmath}
\usepackage{amssymb}
\usepackage[all]{xy}
\ifx\hypersetup\undefined
  \AtBeginDocument{%
    \hypersetup{unicode=true,pdfusetitle,
 bookmarks=true,bookmarksnumbered=false,bookmarksopen=false,
 breaklinks=false,pdfborder={0 0 1},backref=false,colorlinks=true}
  }
\else
  \hypersetup{unicode=true,pdfusetitle,
 bookmarks=true,bookmarksnumbered=false,bookmarksopen=false,
 breaklinks=false,pdfborder={0 0 1},backref=false,colorlinks=true}
\fi

\makeatletter

%%%%%%%%%%%%%%%%%%%%%%%%%%%%%% LyX specific LaTeX commands.
%% Because html converters don't know tabularnewline
\providecommand{\tabularnewline}{\\}

%%%%%%%%%%%%%%%%%%%%%%%%%%%%%% Textclass specific LaTeX commands.
 % this default might be overridden by plain title style
 \newcommand\makebeamertitle{\frame{\maketitle}}%
 % (ERT) argument for the TOC
 \AtBeginDocument{%
   \let\origtableofcontents=\tableofcontents
   \def\tableofcontents{\@ifnextchar[{\origtableofcontents}{\gobbletableofcontents}}
   \def\gobbletableofcontents#1{\origtableofcontents}
 }
 \newenvironment{lyxcode}
   {\par\begin{list}{}{
     \setlength{\rightmargin}{\leftmargin}
     \setlength{\listparindent}{0pt}% needed for AMS classes
     \raggedright
     \setlength{\itemsep}{0pt}
     \setlength{\parsep}{0pt}
     \normalfont\ttfamily}%
    \def\{{\char`\{}
    \def\}{\char`\}}
    \def\textasciitilde{\char`\~}
    \item[]}
   {\end{list}}

%%%%%%%%%%%%%%%%%%%%%%%%%%%%%% User specified LaTeX commands.
\usetheme[secheader]{Boadilla}
\usecolortheme{seahorse}
\title[Chapter 7: Functor-lifted computations II]{Chapter 7: Computations lifted to a functor context II. Monads}
\subtitle{Part 2: Laws and structure of semimonads}
\author{Sergei Winitzki}
\date{2018-03-11}
\institute[ABTB]{Academy by the Bay}
\setbeamertemplate{headline}{} % disable headline at top
\setbeamertemplate{navigation symbols}{} % disable navigation bar at bottom
\usepackage[all]{xy}
\makeatletter
% Macros to assist LyX with XYpic when using scaling.
\newcommand{\xyScaleX}[1]{%
\makeatletter
\xydef@\xymatrixcolsep@{#1}
\makeatother
} % end of \xyScaleX
\makeatletter
\newcommand{\xyScaleY}[1]{%
\makeatletter
\xydef@\xymatrixrowsep@{#1}
\makeatother
} % end of \xyScaleY

\makeatother

\begin{document}
\frame{\titlepage}
\begin{frame}{Semimonad laws I: The intuitions}

What properties of functor block programs do we expect to have?
\begin{itemize}
\item In \texttt{\textcolor{blue}{\footnotesize{}x $\leftarrow$ c}}, the
value of \texttt{\textcolor{blue}{\footnotesize{}x}} will \emph{go
over items} held in container \texttt{\textcolor{blue}{\footnotesize{}c}} 
\item Manipulating items in container is followed by a generator:
\end{itemize}
\texttt{\textcolor{blue}{\footnotesize{}}}%
\begin{minipage}[c][1\totalheight][t]{0.49\columnwidth}%
\begin{lyxcode}
\textcolor{blue}{\footnotesize{}x~$\leftarrow$~cont1}{\footnotesize \par}

\textcolor{blue}{\footnotesize{}y~=~f(x)}{\footnotesize \par}

\textcolor{blue}{\footnotesize{}z~$\leftarrow$~cont2(y)}{\footnotesize \par}
\end{lyxcode}
%
\end{minipage}\texttt{\textcolor{blue}{\footnotesize{}\hfill{}}}%
\begin{minipage}[c][1\totalheight][t]{0.4\columnwidth}%
\begin{lyxcode}
\textcolor{blue}{\footnotesize{}y~$\leftarrow$~cont1}{\footnotesize \par}

\textcolor{blue}{\footnotesize{}~~~~~~.map(x~$\Rightarrow$~f(x))}{\footnotesize \par}

\textcolor{blue}{\footnotesize{}z~$\leftarrow$~cont2(y)}{\footnotesize \par}
\end{lyxcode}
%
\end{minipage}\texttt{\textcolor{blue}{\footnotesize{}\hfill{}\medskip{}
}}{\footnotesize \par}

\texttt{\textcolor{blue}{\footnotesize{}cont1.flatMap(x $\Rightarrow$
cont2(f(x))) = cont1.map(f).flatMap(y $\Rightarrow$ cont2(y))}} 
\begin{itemize}
\item Manipulating items in container is preceded by a generator:
\end{itemize}
\texttt{\textcolor{blue}{\footnotesize{}}}%
\begin{minipage}[c][1\totalheight][t]{0.49\columnwidth}%
\begin{lyxcode}
\textcolor{blue}{\footnotesize{}x~$\leftarrow$~cont1}{\footnotesize \par}

\textcolor{blue}{\footnotesize{}y~$\leftarrow$~cont2(x)}{\footnotesize \par}

\textcolor{blue}{\footnotesize{}z~=~f(y)}{\footnotesize \par}
\end{lyxcode}
%
\end{minipage}\texttt{\textcolor{blue}{\footnotesize{}\hfill{}}}%
\begin{minipage}[c][1\totalheight][t]{0.49\columnwidth}%
\begin{lyxcode}
\textcolor{blue}{\footnotesize{}x~$\leftarrow$~cont1}{\footnotesize \par}

\textcolor{blue}{\footnotesize{}z~$\leftarrow$~cont2(x)}{\footnotesize \par}

\textcolor{blue}{\footnotesize{}~~~~~~~.map(f)}{\footnotesize \par}
\end{lyxcode}
%
\end{minipage}\texttt{\textcolor{blue}{\footnotesize{}\hfill{}\medskip{}
cont1.flatMap(cont2).map(f)}} \texttt{\textcolor{blue}{\footnotesize{}=
cont1.flatMap(x $\Rightarrow$ cont2(x).map(f))}} 
\begin{itemize}
\item After \texttt{\textcolor{blue}{\footnotesize{}x $\leftarrow$ cont}},
further computations will use \emph{all those} \texttt{\textcolor{blue}{\footnotesize{}x}} 
\end{itemize}
\texttt{\textcolor{blue}{\footnotesize{}}}%
\begin{minipage}[c][1\totalheight][t]{0.49\columnwidth}%
\begin{lyxcode}
\textcolor{blue}{\footnotesize{}x~$\leftarrow$~cont}{\footnotesize \par}

\textcolor{blue}{\footnotesize{}y~$\leftarrow$~p(x)}{\footnotesize \par}

\textcolor{blue}{\footnotesize{}z~$\leftarrow$~cont2(y)}{\footnotesize \par}
\end{lyxcode}
%
\end{minipage}\texttt{\textcolor{blue}{\footnotesize{}\hfill{}}}%
\begin{minipage}[c][1\totalheight][t]{0.49\columnwidth}%
\begin{lyxcode}
\textcolor{blue}{\footnotesize{}y~$\leftarrow$~for~\{~x~$\leftarrow$~cont}{\footnotesize \par}

\textcolor{blue}{\footnotesize{}~~~~~~~~~~~yy~$\leftarrow$~p(x)~\}~yield~yy}{\footnotesize \par}

\textcolor{blue}{\footnotesize{}z~$\leftarrow$~cont2(y)}{\footnotesize \par}
\end{lyxcode}
%
\end{minipage}\texttt{\textcolor{blue}{\footnotesize{}\hfill{}\medskip{}
cont.flatMap(x $\Rightarrow$ p(x).flatMap(cont2)) = cont.flatMap(p).flatMap(cont2)}} 
\end{frame}

\begin{frame}{Semimonad laws II: The laws for \texttt{\textcolor{blue}{\footnotesize{}flatMap}} }

To get a more concise notation, use {\footnotesize{}$\text{flm}$}
instead of \texttt{\textcolor{blue}{\footnotesize{}flatMap}} 

A \textbf{semimonad} $S^{A}$ has {\footnotesize{}$\text{flm}^{\left[S,A,B\right]}:\left(A\Rightarrow S^{B}\right)\Rightarrow S^{A}\Rightarrow S^{B}$}
with 3 laws:
\begin{enumerate}
\item {\footnotesize{}$\text{flm}\,(f^{A\Rightarrow B}\circ g^{B\Rightarrow S^{C}})=\text{fmap}\,f\circ\text{flm}\,g$}
{\footnotesize{}(naturality in $A$)} {\footnotesize{}
\[
\xymatrix{\xyScaleY{0.2pc}\xyScaleX{3pc} & S^{B}\ar[rd]\sp(0.5){\ \text{flm}\,g^{B\Rightarrow S^{C}}}\\
S^{A}\ar[ru]\sp(0.5){\text{fmap}\,f^{A\Rightarrow B}\ }\ar[rr]\sb(0.5){\text{flm}\,(f^{A\Rightarrow B}\circ\,g^{B\Rightarrow S^{C}})\,} &  & S^{C}
}
\]
}{\footnotesize \par}
\item {\footnotesize{}$\text{flm}\,\big(f^{A\Rightarrow S^{B}}\circ\text{fmap}\,g^{B\Rightarrow C}\big)=\text{flm}\,f\circ\text{fmap}\,g$}
{\footnotesize{}(naturality in $B$)} {\footnotesize{}
\[
\xymatrix{\xyScaleY{0.2pc}\xyScaleX{3pc} & S^{B}\ar[rd]\sp(0.5){\ \text{fmap}\,g^{B\Rightarrow C}}\\
S^{A}\ar[ru]\sp(0.5){\text{flm}\,f^{A\Rightarrow S^{B}}\ }\ar[rr]\sb(0.5){\text{flm}\,(f^{A\Rightarrow S^{B}}\circ\,\text{fmap}\,g^{B\Rightarrow C})\,} &  & S^{C}
}
\]
}{\footnotesize \par}
\item {\footnotesize{}$\text{flm}\,\big(f^{A\Rightarrow S^{B}}\circ\text{flm}\,g^{B\Rightarrow S^{C}}\big)=\text{flm}\,f\circ\text{flm}\,g$}
{\footnotesize{}(associativity)} {\footnotesize{}
\[
\xymatrix{\xyScaleY{0.2pc}\xyScaleX{3pc} & S^{B}\ar[rd]\sp(0.5){\ \text{flm}\,g^{B\Rightarrow S^{C}}}\\
S^{A}\ar[ru]\sp(0.5){\text{flm}\,f^{A\Rightarrow S^{B}}\ }\ar[rr]\sb(0.5){\text{flm}\,\big(f^{A\Rightarrow S^{B}}\circ\,\text{flm}\,g^{B\Rightarrow S^{C}}\big)\,} &  & S^{C}
}
\]
}{\footnotesize \par}
\end{enumerate}
Is there a shorter formulation of the laws?
\end{frame}

\begin{frame}{Semimonad laws III: The laws for \texttt{\textcolor{blue}{\footnotesize{}flatten}} }

The methods \texttt{\textcolor{blue}{\footnotesize{}flatten}} (denoted
by {\footnotesize{}$\text{ftn}$}) and \texttt{\textcolor{blue}{\footnotesize{}flatMap}}
are equivalent:\texttt{\textcolor{blue}{\footnotesize{} }}%
\begin{minipage}[c][1\totalheight][t]{0.4\columnwidth}%
{\footnotesize{}
\begin{align*}
\text{ftn}^{\left[S,A\right]}:S^{S^{A}}\Rightarrow S^{A} & =\text{flm}^{\left[S,S^{A},A\right]}(m^{S^{A}}\Rightarrow m)\\
\text{flm}\,\big(f^{A\Rightarrow S^{B}}\big) & =\text{fmap}\,f\circ\text{ftn}
\end{align*}
}%
\end{minipage}\texttt{\textcolor{blue}{\footnotesize{}\hfill{}}}%
\begin{minipage}[c][1\totalheight][t]{0.4\columnwidth}%
{\footnotesize{}
\[
\xymatrix{\xyScaleY{0.2pc}\xyScaleX{3pc} & S^{S^{B}}\ar[rd]\sp(0.5){\ \text{ftn}\ }\\
S^{A}\ar[ru]\sp(0.5){\text{fmap}\,f^{A\Rightarrow S^{B}}\ }\ar[rr]\sb(0.5){\text{flm}\,\big(f^{A\Rightarrow S^{B}}\big)\,} &  & S^{B}
}
\]
}%
\end{minipage}\texttt{\textcolor{blue}{\footnotesize{}\  \  \ \hfill{}}}{\footnotesize \par}

It turns out that \texttt{\textcolor{blue}{\footnotesize{}flatten}}
has only 2 laws:
\begin{enumerate}
\item {\footnotesize{}$\text{fmap}\big(\text{fmap}\,f^{A\Rightarrow B}\big)\circ\text{ftn}^{\left[S,B\right]}=\text{ftn}^{\left[S,A\right]}\circ\text{fmap}\,f$}
{\footnotesize{}(naturality)
\[
\xymatrix{\xyScaleY{0.2pc}\xyScaleX{5pc} & S^{S^{B}}\ar[rd]\sp(0.5){\ \text{ftn}^{\left[S,B\right]}}\\
S^{S^{A}}\ar[ru]\sp(0.5){\text{fmap}\,\big(\text{fmap}\,f^{A\Rightarrow B}\big)\ \ }\ar[rd]\sb(0.5){\text{ftn}^{\left[S,A\right]}\,} &  & S^{B}\\
 & S^{A}\ar[ru]\sb(0.5){\text{fmap}\,f^{A\Rightarrow B}}
}
\]
}{\footnotesize \par}
\item {\footnotesize{}$\text{fmap}\,(\text{ftn}^{\left[S,A\right]})\circ\text{ftn}^{\left[S,A\right]}=\text{ftn}^{[S,S^{A}]}\circ\text{ftn}^{\left[S,A\right]}$}
{\footnotesize{}(associativity) 
\[
\xymatrix{\xyScaleY{0.2pc}\xyScaleX{5pc} & S^{S^{A}}\ar[rd]\sp(0.5){\ \text{ftn}^{\left[S,A\right]}}\\
S^{S^{S^{A}}}\ar[ru]\sp(0.5){\text{fmap}\,(\text{ftn}^{\left[S,A\right]})\ }\ar[rd]\sb(0.5){\text{ftn}^{[S,S^{A}]}\,} &  & S^{A}\\
 & S^{S^{A}}\ar[ru]\sb(0.5){\text{ftn}^{\left[S,A\right]}}
}
\]
}{\footnotesize \par}
\end{enumerate}
\end{frame}

\begin{frame}{Equivalence of a natural transformation and a ``lifting''}
\begin{itemize}
\item Equivalence of {\footnotesize{}$\text{flm}$} and {\footnotesize{}$\text{ftn}$}:
{\footnotesize{}$\text{ftn}=\text{flm}\left(\text{id}\right)$; $\text{flm}\,f=\text{fmap}\,f\circ\text{ftn}$} 
\item We saw this before: {\footnotesize{}$\text{deflate}=\text{fmapOpt}\left(\text{id}\right)$};
{\footnotesize{}$\text{fmapOpt}\,f=\text{fmap}\:f\circ\text{deflate}$} 
\begin{itemize}
\item Is there a general pattern where two such functions are equivalent?
\end{itemize}
\item Let {\footnotesize{}$\text{tr}:F^{G^{A}}\Rightarrow F^{A}$ }be a
natural transformation ($F$ and $G$ are functors)
\item Define {\footnotesize{}$\text{ftr}:\left(A\Rightarrow G^{B}\right)\Rightarrow F^{A}\Rightarrow F^{B}$}
by {\footnotesize{}$\text{ftr}\,f=\text{fmap}\,f\circ\text{tr}$} 
\item It follows that {\footnotesize{}$\text{tr}=\text{ftr}\left(\text{id}\right)$},
and we have equivalence between {\footnotesize{}$\text{tr}$} and
{\footnotesize{}$\text{ftr}$}:\texttt{\textcolor{blue}{\footnotesize{} }}%
\begin{minipage}[c][1\totalheight][t]{0.4\columnwidth}%
{\footnotesize{}
\begin{align*}
\text{tr}:F^{G^{A}}\Rightarrow F^{A} & =\text{ftr}(m^{G^{A}}\Rightarrow m)\\
\text{ftr}\,\big(f^{A\Rightarrow G^{B}}\big) & =\text{fmap}\,f\circ\text{tr}
\end{align*}
}%
\end{minipage}\texttt{\textcolor{blue}{\footnotesize{}\hfill{}}}%
\begin{minipage}[c][1\totalheight][t]{0.4\columnwidth}%
{\footnotesize{}
\[
\xymatrix{\xyScaleY{0.2pc}\xyScaleX{3pc} & F^{G^{B}}\ar[rd]\sp(0.5){\ \text{tr}\ }\\
F^{A}\ar[ru]\sp(0.5){\text{fmap}\,f^{A\Rightarrow G^{B}}\ }\ar[rr]\sb(0.5){\text{ftr}\,\big(f^{A\Rightarrow G^{B}}\big)\,} &  & F^{B}
}
\]
}%
\end{minipage}\texttt{\textcolor{blue}{\footnotesize{}\  \  \ \hfill{}}}{\footnotesize \par}
\item An automatic law for {\footnotesize{}$\text{ftr}$} (``naturality
in $A$'') follows from the definition: {\footnotesize{}$\text{fmap}\,g\circ\text{ftr}\,f=\text{fmap}\,g\circ\text{fmap}\,f\circ\text{tr}=\text{fmap}\left(g\circ f\right)\circ\text{tr}=\text{ftr}\left(g\circ f\right)$} 
\begin{itemize}
\item This is why {\footnotesize{}$\text{tr}$} has \emph{one law fewer}
than {\footnotesize{}$\text{ftr}$}{\footnotesize \par}
\end{itemize}
\item To demonstrate equivalence in the direction {\footnotesize{}$\text{ftr}\rightarrow\text{tr}$}:
start with an arbitrary {\footnotesize{}$\text{ftr}$} satisfying
``naturality in $A$'', then obtain {\footnotesize{}$\text{tr}=\text{ftr}\left(\text{id}\right)$}
from it, then verify {\footnotesize{}$\text{ftr}\,f=\text{fmap}\,f\circ\text{tr}$}
with that {\footnotesize{}$\text{tr}$}: {\footnotesize{}$\text{fmap}\,f\circ\text{ftr}\left(\text{id}\right)=\text{ftr}\left(f\circ id\right)=\text{ftr}\,f$}{\footnotesize \par}
\end{itemize}
\end{frame}

\begin{frame}{Semimonad laws IV: Deriving the laws for \texttt{\textcolor{blue}{\footnotesize{}flatten}} }

Denote for brevity $q^{\uparrow}\equiv\text{fmap}^{\left[S\right]}\,q$
for any function $q$

Express $\text{flm}\,f=f^{\uparrow}\circ\text{ftn}$ and substitute
that into $\text{flm}$'s 3 laws:
\begin{enumerate}
\item {\footnotesize{}$\text{flm}\left(f\circ g\right)=f^{\uparrow}\circ\text{flm}\,g$}
gives {\footnotesize{}$\left(f\circ g\right)^{\uparrow}\circ\text{ftn}=f^{\uparrow}\circ g^{\uparrow}\circ\text{ftn}$}\\
\textendash{} this law holds automatically due to functor composition
law
\item {\footnotesize{}$\text{flm}\left(f\circ g^{\uparrow}\right)=\text{flm}\,f\circ g^{\uparrow}$}
gives {\footnotesize{}$\left(f\circ h\right)^{\uparrow}\circ\text{ftn}=f^{\uparrow}\circ\text{ftn}\circ h$};\\
using the functor composition law, we reduce this to\\
{\footnotesize{}$h^{\uparrow}\circ\text{ftn}=\text{ftn}\circ h$}
\textendash{} this is the naturality law
\item {\footnotesize{}$\text{flm}\left(f\circ\text{flm}\,g\right)=\text{flm}\,f\circ\text{flm}\,g$
}with functor composition law gives{\footnotesize{} $f^{\uparrow}\circ g^{\uparrow\uparrow}\circ\text{ftn}^{\uparrow}\circ\text{ftn}=f^{\uparrow}\circ\text{ftn}\circ g^{\uparrow}\circ\text{ftn}$;}
using {\footnotesize{}$\text{ftn}$}'s naturality and omitting the
common factor{\footnotesize{} $f^{\uparrow}\circ g^{\uparrow\uparrow}$},
we get{\footnotesize{} $\text{ftn}^{\uparrow}\circ\text{ftn}=\text{ftn}\circ\text{ftn}$}
\textendash{} associativity law
\end{enumerate}
\begin{itemize}
\item \texttt{\textcolor{blue}{\footnotesize{}flatten}} has the simplest
type signature \emph{and} the fewest laws
\item It is usually easy to check naturality!
\begin{itemize}
\item \textbf{Parametricity theorem}: Any \emph{pure, fully parametric}
code for a function of type $F^{A}\Rightarrow G^{A}$ will implement
a natural transformation
\end{itemize}
\item Checking \texttt{\textcolor{blue}{\footnotesize{}flatten}}'s associativity
needs \emph{a lot} more work!
\end{itemize}
The \texttt{\textcolor{blue}{\footnotesize{}cats}} library has a \texttt{\textcolor{blue}{\footnotesize{}FlatMap}}
type class, defining \texttt{\textcolor{blue}{\footnotesize{}flatten}}
via \texttt{\textcolor{blue}{\footnotesize{}flatMap}} 
\end{frame}

\begin{frame}{Checking the associativity law for standard monads}
\begin{itemize}
\item Implement \texttt{\textcolor{blue}{\footnotesize{}flatten}} for these
functors and check the laws (see code):
\begin{itemize}
\item \texttt{\textcolor{blue}{\footnotesize{}Option}} monad: $F^{A}\equiv1+A$;
$\text{ftn}:1+\left(1+A\right)\Rightarrow1+A$
\item \texttt{\textcolor{blue}{\footnotesize{}Either}} monad: $F^{A}\equiv Z+A$;
$\text{ftn}:Z+\left(Z+A\right)\Rightarrow Z+A$
\item \texttt{\textcolor{blue}{\footnotesize{}List}} monad: $F^{A}\equiv\text{List}^{A}$;
$\text{ftn}:\text{List}^{\text{List}^{A}}\Rightarrow\text{List}^{A}$
\item Writer monad: $F^{A}\equiv A\times W$; $\text{ftn}:\left(A\times W\right)\times W\Rightarrow A\times W$
\item Reader monad: $F^{A}\equiv R\Rightarrow A$; $\text{ftn}:\left(R\Rightarrow\left(R\Rightarrow A\right)\right)\Rightarrow R\Rightarrow A$
\item State: $F^{A}\equiv S\Rightarrow A\times S$; $\text{ftn}:\left(S\Rightarrow\left(S\Rightarrow A\times S\right)\times S\right)\Rightarrow S\Rightarrow A\times S$
\item Continuation monad: $F^{A}\equiv\left(A\Rightarrow R\right)\Rightarrow R$;
$\text{ftn}:\left(\left(\left(\left(A\Rightarrow R\right)\Rightarrow R\right)\Rightarrow R\right)\Rightarrow R\right)\Rightarrow\left(A\Rightarrow R\right)\Rightarrow R$
\end{itemize}
\item Code implementing these \texttt{\textcolor{blue}{\footnotesize{}flatten}}
functions is \emph{fully parametric} in $A$
\begin{itemize}
\item Naturality of these functions follows from parametricity theorem
\item Associativity needs to be checked for each monad!
\end{itemize}
\item Example of a useful semimonad that is \emph{not} a full monad:
\begin{itemize}
\item $F^{A}\equiv A\times V\times W$; $\text{ftn}\left(\left(a\times v_{1}\times w_{1}\right)\times v_{2}\times w_{2}\right)=a\times v_{1}\times w_{2}$
\end{itemize}
\item Examples of \emph{non-associative} (i.e.\ wrong) implementations
of \texttt{\textcolor{blue}{\footnotesize{}flatten}}:
\begin{itemize}
\item $F^{A}\equiv A\times W\times W$; $\text{ftn}\left(\left(a\times v_{1}\times v_{2}\right)\times w_{1}\times w_{2}\right)=a\times w_{2}\times w_{1}$
\item $F^{A}\equiv\text{List}^{A}$, but \texttt{\textcolor{blue}{\footnotesize{}flatten}}
concatenates the nested lists in reverse order
\end{itemize}
\end{itemize}
\end{frame}

\begin{frame}{Motivation for monads}

\begin{itemize}
\item Monads represent values with a ``special computational context''
\item Specific monads will have methods to create various contexts
\item Monadic composition will ``combine'' the contexts associatively
\item It is generally useful to have an ``empty context'' available
\[
\text{pure}:A\Rightarrow M^{A}
\]
\item Combining empty context with another context works as a no-op
\item Empty context is followed by a generator:
\end{itemize}
\texttt{\textcolor{blue}{\footnotesize{}}}%
\begin{minipage}[c][1\totalheight][t]{0.49\columnwidth}%
\begin{lyxcode}
\textcolor{blue}{\footnotesize{}y~$\leftarrow$~pure(x)}{\footnotesize \par}

\textcolor{blue}{\footnotesize{}z~$\leftarrow$~cont(y)}{\footnotesize \par}
\end{lyxcode}
%
\end{minipage}\texttt{\textcolor{blue}{\footnotesize{}\hfill{}}}%
\begin{minipage}[c][1\totalheight][t]{0.4\columnwidth}%
\begin{lyxcode}
\textcolor{blue}{\footnotesize{}y~=~x}{\footnotesize \par}

\textcolor{blue}{\footnotesize{}z~$\leftarrow$~cont(y)}{\footnotesize \par}
\end{lyxcode}
%
\end{minipage}\texttt{\textcolor{blue}{\footnotesize{}\hfill{}\medskip{}
}}{\footnotesize \par}

\texttt{\textcolor{blue}{\footnotesize{}pure(x).flatMap(y $\Rightarrow$
cont(y)) = cont(x)}}$\quad\quad\text{pure}\circ\text{flm}\,f=f$ \textcolor{gray}{\textendash{}
left identity}
\begin{itemize}
\item Empty context is preceded by a generator:
\end{itemize}
\texttt{\textcolor{blue}{\footnotesize{}}}%
\begin{minipage}[c][1\totalheight][t]{0.49\columnwidth}%
\begin{lyxcode}
\textcolor{blue}{\footnotesize{}x~$\leftarrow$~cont}{\footnotesize \par}

\textcolor{blue}{\footnotesize{}y~$\leftarrow$~pure(x)}{\footnotesize \par}
\end{lyxcode}
%
\end{minipage}\texttt{\textcolor{blue}{\footnotesize{}\hfill{}}}%
\begin{minipage}[c][1\totalheight][t]{0.49\columnwidth}%
\begin{lyxcode}
\textcolor{blue}{\footnotesize{}x~$\leftarrow$~cont}{\footnotesize \par}

\textcolor{blue}{\footnotesize{}y~=~x}{\footnotesize \par}
\end{lyxcode}
%
\end{minipage}\texttt{\textcolor{blue}{\footnotesize{}\hfill{}\medskip{}
cont.flatMap(x $\Rightarrow$ pure(x))}} \texttt{\textcolor{blue}{\footnotesize{}=
cont}} $\quad\quad\quad\text{flm}\left(\text{pure}\right)=\text{id}$
\textcolor{gray}{\textendash{} right identity}
\end{frame}

\begin{frame}{The monad laws formulated in terms of \texttt{\textcolor{blue}{\footnotesize{}pure}}
and \texttt{\textcolor{blue}{\footnotesize{}flatten}} }
\begin{itemize}
\item Naturality law for \texttt{\textcolor{blue}{\footnotesize{}pure}}:
$f\circ\text{pure}=\text{pure}\circ f^{\uparrow}$
\[
\xymatrix{\xyScaleY{0.2pc}\xyScaleX{5pc} & B\ar[rd]\sp(0.5){\ \text{pure}^{\left[S,B\right]}}\\
A\ar[ru]\sp(0.5){f^{A\Rightarrow B}\ }\ar[rd]\sb(0.5){\text{pure}^{[S,A]}\,} &  & S^{B}\\
 & S^{A}\ar[ru]\sb(0.5){\text{fmap}\,f^{A\Rightarrow B}}
}
\]
\item Left identity: $\text{pure}\circ\text{flm}\,f=\text{pure}\circ f^{\uparrow}\circ\text{ftn}=f\circ\text{pure}\circ\text{ftn}=f$
requires that $\text{pure}\circ\text{ftn}=\text{id}$ (both sides
applied to $S^{A}$)
\[
\xymatrix{\xyScaleY{0.2pc}\xyScaleX{3pc} & S^{S^{A}}\ar[rd]\sp(0.5){\ \text{ftn}\ }\\
S^{A}\ar[ru]\sp(0.5){\text{pure}^{[S,S^{A}]}\ }\ar[rr]\sb(0.5){\text{id}} &  & S^{A}
}
\]
\item Right identity: $\text{flm}\left(\text{pure}\right)=\text{pure}^{\uparrow}\circ\text{ftn}=\text{id}$
\[
\xymatrix{\xyScaleY{0.2pc}\xyScaleX{3pc} & S^{S^{A}}\ar[rd]\sp(0.5){\ \text{ftn}\ }\\
S^{A}\ar[ru]\sp(0.5){\text{fmap}(\text{pure}^{[S,A]})\quad}\ar[rr]\sb(0.5){\text{id}} &  & S^{A}
}
\]
\end{itemize}
\end{frame}

\begin{frame}{Formulating laws via Kleisli functions}
\begin{itemize}
\item Recall: we formulated the laws of filterables via \texttt{\textcolor{blue}{\footnotesize{}fmapOpt}} 
\begin{itemize}
\item $\text{fmapOpt}:\left(A\Rightarrow1+B\right)\Rightarrow S^{A}\Rightarrow S^{B}$
\end{itemize}
\item And then we had to compose functions of types $A\Rightarrow1+B$ with
$\diamond_{\text{Opt}}$
\item Here we have{\small{} $\text{flm}:\left(A\Rightarrow S^{B}\right)\Rightarrow S^{A}\Rightarrow S^{B}$}
instead of \texttt{\textcolor{blue}{\footnotesize{}fmapOpt}} 
\item Can we compose \textbf{Kleisli functions} with ``twisted'' types
$A\Rightarrow S^{B}$?
\item Use $\text{flm}$ to define \textbf{Kleisli composition}: $f^{A\Rightarrow S^{B}}\diamond g^{B\Rightarrow S^{C}}\equiv f\circ\text{flm}\,g$
\item Define \textbf{Kleisli identity} $\text{id}_{\diamond}$ of type $A\Rightarrow S^{A}$
as $\text{id}_{\diamond}\equiv\text{pure}$
\item Composition law: $\text{flm}\left(f\diamond g\right)=\text{flm}\,f\circ\text{flm}\,g$
(same as for \texttt{\textcolor{blue}{\footnotesize{}fmapOpt}})
\begin{itemize}
\item Shows that \texttt{\textcolor{blue}{\footnotesize{}flatMap}} is a
``lifting'' of $A\Rightarrow S^{B}$ to $S^{A}\Rightarrow S^{B}$
\end{itemize}
\item These laws are similar to functor ``lifting'' laws...
\begin{itemize}
\item except that $\diamond$ is used for composing Kleisli functions
\end{itemize}
\item What are the properties of $\diamond$?
\begin{itemize}
\item Exactly similar to the properties of function composition $f\circ g$
\end{itemize}
\end{itemize}
Reformulate $\text{flm}$'s laws in terms of the $\diamond$ operation:
\begin{itemize}
\item $\text{flm}$'s left and right identity laws: $\text{pure}\diamond f=f$
and $f\diamond\text{pure}=f$
\item Associativity law: $\left(f\diamond g\right)\diamond h=f\diamond\left(g\diamond h\right)$
\begin{itemize}
\item Follows from the $\text{flm}$ law: $f\circ\text{flm}\left(g\circ\text{flm}h\right)=f\circ\text{flm}\,g\circ\text{flm}\,h$
\end{itemize}
\end{itemize}
\end{frame}

\begin{frame}{From Kleisli back to \texttt{\textcolor{blue}{\footnotesize{}flatMap}} }

Compare different ``liftings'' seen so far:
\begin{center}
\begin{tabular}{|c|c|c|c|}
\hline 
\textbf{Category} & \textbf{Function type} & \textbf{Identity} & \textbf{Composition}\tabularnewline
\hline 
\hline 
plain functions & $A\Rightarrow B$ & $\text{id}:A\Rightarrow A$ & $f^{A\Rightarrow B}\circ g^{B\Rightarrow C}$\tabularnewline
\hline 
lifted to $F$ & $F^{A}\Rightarrow F^{B}$ & $\text{id}:F^{A}\Rightarrow F^{A}$ & $f^{F^{A}\Rightarrow F^{B}}\circ g^{F^{B}\Rightarrow F^{C}}$\tabularnewline
\hline 
Kleisli over $F$ & $A\Rightarrow F^{B}$ & $\text{pure}:A\Rightarrow F^{A}$ & $f^{A\Rightarrow F^{B}}\diamond g^{B\Rightarrow F^{C}}$\tabularnewline
\hline 
\end{tabular}
\par\end{center}

\textbf{Category} axioms: identity and associativity for composition

General \textbf{functor}: a ``lifting'' maps functions from one
category to another
\begin{itemize}
\item Functor laws: ``lifting'' must preserve identity and composition
\end{itemize}
Reformulate \texttt{\textcolor{blue}{\footnotesize{}map}} and \texttt{\textcolor{blue}{\footnotesize{}flatMap}}
in terms of the $\diamond$ operation:
\begin{itemize}
\item Define \texttt{\textcolor{blue}{\footnotesize{}flatMap}} through Kleisli
composition:{\small{} $\text{flm}\,f^{A\Rightarrow S^{B}}\equiv\text{id}^{S^{A}\Rightarrow S^{A}}\diamond f$}{\small \par}
\item Define \texttt{\textcolor{blue}{\footnotesize{}flatten}} through Kleisli:{\small{}
$\text{ftn}\equiv\text{id}^{S^{S^{A}}\Rightarrow S^{S^{A}}}\diamond\text{id}^{S^{A}\Rightarrow S^{A}}$}{\small \par}
\item Express \texttt{\textcolor{blue}{\footnotesize{}fmap}} through Kleisli:
$\text{fmap}\,f\equiv\left(\text{fmap}\,\text{id}\right)\diamond\left(f\circ\text{pure}\right)$
\item Need two additional laws to connect $\diamond$ and $\circ$:
\begin{itemize}
\item Left naturality: {\small{}$f^{A\Rightarrow B}\circ g^{B\Rightarrow S^{C}}=\left(f\circ\text{pure}\right)\diamond g$}{\small \par}
\item Right naturality: {\small{}$f^{A\Rightarrow B}\circ\text{fmap}\,g^{B\Rightarrow S^{C}}=f\diamond\left(g\circ\text{pure}\right)$}{\small \par}
\begin{itemize}
\item With these laws, monad laws follow from category axioms for Kleisli
\end{itemize}
\end{itemize}
\end{itemize}
\end{frame}

\begin{frame}{Structure of semigroups and monoids}

\begin{itemize}
\item Semimonad contexts are combined associatively, as in a semigroup
\item A full monad includes an ``empty'' context, i.e.\ the identity
element
\item Semigroup with an identity element is a monoid
\end{itemize}
Some constructions of semigroups and monoids:
\begin{enumerate}
\item Any type $Z$ is a semigroup with operation $z_{1}\circledast z_{2}=z_{1}$
(or $z_{2}$)
\item $1+S$ is a monoid if $S$ is (at least) a semigroup
\item $\text{List}^{A}$ is a monoid (for any type $A$), also $\text{Seq}^{A}$
etc.
\item The function type $A\Rightarrow A$ is a monoid (for any type $A$)
\begin{itemize}
\item The operation $f\circledast g$ is either $f\circ g$ or $g\circ f$
\end{itemize}
\item Any totally ordered type is a monoid, with $\circledast$ defined
as $\max$ or $\min$
\item $S_{1}\times S_{2}$ is a semigroup (monoid) if $S_{1}$, $S_{2}$
are semigroups (monoids)
\item $S\times P$ is a semigroup (monoid) if $S$ is a semigroup (monoid)
such that $S$ acts on $P$. (``Twisted product.'') Example: $\left(A\Rightarrow A\right)\times A$
\begin{itemize}
\item The ``action'' is $a:S\Rightarrow P\Rightarrow P$ such that $a(s_{1})\circ a(s_{2})=a(s_{1}\circledast s_{2})$.
\end{itemize}
\item $Z\Rightarrow S$ is a semigroup/monoid, for any $Z$, if $S$ is
a semigroup/monoid
\end{enumerate}
\begin{itemize}
\item There are other examples: $\text{Int}$, $\text{String}$, $\text{Set}^{A}$,
Akka routes, ...
\item Non-examples: trees; $S_{1}+S_{2}$ where $S_{1,2}$ are different
monoids
\end{itemize}
\end{frame}

\begin{frame}{Structure of (semi)monads}


\framesubtitle{How to recognize a (semi)monad by its type? Open question!}

Intuition from \texttt{\textcolor{blue}{\footnotesize{}flatten}}:
reshuffle data in $F^{F^{A}}$ to fit into $F^{A}$

Some constructions of exponential-polynomial semimonads:
\begin{enumerate}
\item $F^{A}\equiv Z$ (constant functor) for a fixed type $Z$
\begin{itemize}
\item For a full monad, need to choose $Z=1$ 
\end{itemize}
\item $F^{A}\equiv A\times G^{A}$ for any functor $G^{A}$ (a full monad
only if $G^{A}\equiv1$)
\item $F^{A}\equiv Z+A\times W$ for a fixed type $Z$ and a semigroup $W$
\begin{itemize}
\item For a full monad, need $W$ to be a monoid
\end{itemize}
\item $F^{A}\equiv G^{Z+A\times W}$ if $Z+A\times W$ is a (semi)monad
\item $F^{A}\equiv G^{A}\times H^{A}$ for any (semi)monads $G^{A}$ and
$H^{A}$
\item $F^{A}\equiv A+G^{A}$ for any semimonad $G^{A}$
\item $F^{A}\equiv A+G^{F^{A}}$ (recursive) for any functor $G^{A}$ (\textbf{free
monad} over $G$)
\item $F^{A}\equiv G^{A}+G^{F^{A}}$ (recursive) for any functor $G^{A}$
(semimonad only!)
\item $F^{A}\equiv R\rightarrow G^{A}$ for any (semi)monad $G^{A}$
\item $F^{A}\equiv H^{A}\Rightarrow A\times G^{A}$ for any contrafunctor
$H^{A}$ and functor $G^{A}$
\begin{itemize}
\item For a full monad, need to set $G^{A}\equiv1$
\end{itemize}
\end{enumerate}
\end{frame}

\begin{frame}{Worked examples II: Constructions of filterable functors I}

z
\end{frame}

\begin{frame}{Exercises II}
\begin{enumerate}
\item For an arbitrary monad $M^{A}$, show that the functor $F^{A}\equiv\text{Boolean}\times M^{A}$
can be defined as a semimonad but not a monad.
\item If $W$ and $R$ are arbitrary fixed types, which of the functors
can be made into a semimonad: $F^{A}\equiv W\times\left(R\Rightarrow A\right)$,
$G^{A}=R\Rightarrow\left(W\times A\right)$?
\item Suppose a functor $F^{A}$ has a natural transformation $\text{ex}^{[A]}:F^{A}\Rightarrow A$
that ``extracts the value'' from $F^{A}$. Would $F^{A}$ be a semimonad
if we defined \texttt{\textcolor{blue}{\footnotesize{}flatten}} as
$\text{ftn}=\text{ex}^{[F^{A}]}$ or $\text{ftn}=\text{fmap}\,\text{ex}$?
\item A programmer implemented the \texttt{\textcolor{blue}{\footnotesize{}fmap}}
method for the type constructor $F^{A}\equiv A\times\left(A\Rightarrow Z\right)$
as
\begin{lyxcode}
\textcolor{blue}{\footnotesize{}def~fmap{[}A,B{]}(f:~A$\Rightarrow$B):~((A,~A$\Rightarrow$Z))~$\Rightarrow$~(B,~B$\Rightarrow$Z)~=}{\footnotesize \par}

\textcolor{blue}{\footnotesize{}~~\{~case(a,~az)~$\Rightarrow$~(f(a),~(\_:~B)~$\Rightarrow$~az(a))~\}}{\footnotesize \par}
\end{lyxcode}
Show that this implementation fails to satisfy the functor laws.
\item Implement the \texttt{\textcolor{blue}{\footnotesize{}flatten}} and
\texttt{\textcolor{blue}{\footnotesize{}pure}} methods for the type
constructor $F^{A}\equiv1+A\times A$ (\texttt{\textcolor{blue}{\footnotesize{}type
F{[}A{]} = Option{[}(A, A){]}}}) in at least two different ways, and
show that the monad laws always fail.
\item Implement the monad methods for $F^{A}\equiv\left(Z\Rightarrow1+A\right)\times\text{List}^{A}$
using the known monad constructions.
\end{enumerate}
\end{frame}

\begin{frame}{Exercises II}
\begin{enumerate}
\item []\addtocounter{enumi}{6}(continued from the previous slide)
\item Check the identity laws for monad construction 6, $F^{A}\equiv A+G^{A}$,
when $\text{pure}_{F}$ is defined as $\text{id}^{[A]}+0$ (given
that $G^{A}$ is a monad). Show that the identity laws fail if $\text{pure}_{F}$
is defined as $0+\text{pure}_{G}$.
\item Show that $F^{A}=\left(P\Rightarrow A\right)+\left(Q\Rightarrow A\right)$
is not a semimonad when $P$ and $Q$ are arbitrary, different types.
\end{enumerate}
\end{frame}

\end{document}
