\batchmode
\makeatletter
\def\input@path{{/Users/sergei.winitzki/Code/talks/ftt-fp/}}
\makeatother
\documentclass[english]{beamer}
\usepackage[T1]{fontenc}
\usepackage[latin9]{inputenc}
\setcounter{secnumdepth}{3}
\setcounter{tocdepth}{3}
\usepackage{babel}
\usepackage{amsmath}
\usepackage[all]{xy}
\ifx\hypersetup\undefined
  \AtBeginDocument{%
    \hypersetup{unicode=true,pdfusetitle,
 bookmarks=true,bookmarksnumbered=false,bookmarksopen=false,
 breaklinks=false,pdfborder={0 0 1},backref=false,colorlinks=true}
  }
\else
  \hypersetup{unicode=true,pdfusetitle,
 bookmarks=true,bookmarksnumbered=false,bookmarksopen=false,
 breaklinks=false,pdfborder={0 0 1},backref=false,colorlinks=true}
\fi

\makeatletter
%%%%%%%%%%%%%%%%%%%%%%%%%%%%%% Textclass specific LaTeX commands.
 % this default might be overridden by plain title style
 \newcommand\makebeamertitle{\frame{\maketitle}}%
 % (ERT) argument for the TOC
 \AtBeginDocument{%
   \let\origtableofcontents=\tableofcontents
   \def\tableofcontents{\@ifnextchar[{\origtableofcontents}{\gobbletableofcontents}}
   \def\gobbletableofcontents#1{\origtableofcontents}
 }
 \newenvironment{lyxcode}
   {\par\begin{list}{}{
     \setlength{\rightmargin}{\leftmargin}
     \setlength{\listparindent}{0pt}% needed for AMS classes
     \raggedright
     \setlength{\itemsep}{0pt}
     \setlength{\parsep}{0pt}
     \normalfont\ttfamily}%
    \def\{{\char`\{}
    \def\}{\char`\}}
    \def\textasciitilde{\char`\~}
    \item[]}
   {\end{list}}

%%%%%%%%%%%%%%%%%%%%%%%%%%%%%% User specified LaTeX commands.
\usetheme[secheader]{Boadilla}
\usecolortheme{seahorse}
\title[Chapter 8: Applicative functors]{Chapter 8: Applicative functors and profunctors}
\subtitle{Part 2: Their laws and structure}
\author{Sergei Winitzki}
\date{2018-07-01}
\institute[ABTB]{Academy by the Bay}
\setbeamertemplate{headline}{} % disable headline at top
\setbeamertemplate{navigation symbols}{} % disable navigation bar at bottom
\usepackage[all]{xy}
\makeatletter
% Macros to assist LyX with XYpic when using scaling.
\newcommand{\xyScaleX}[1]{%
\makeatletter
\xydef@\xymatrixcolsep@{#1}
\makeatother
} % end of \xyScaleX
\makeatletter
\newcommand{\xyScaleY}[1]{%
\makeatletter
\xydef@\xymatrixrowsep@{#1}
\makeatother
} % end of \xyScaleY

\makeatother

\begin{document}
\frame{\titlepage}
\begin{frame}{Deriving the \texttt{\textcolor{blue}{\footnotesize{}ap}} operation
from \texttt{\textcolor{blue}{\footnotesize{}map2}} }

\vspace{-0.1cm}Can we avoid having to define $\text{map}_{n}$ separately
for each $n$?
\begin{itemize}
\item Use curried arguments, $\text{fmap}_{2}:(A\Rightarrow B\Rightarrow Z)\Rightarrow F^{A}\Rightarrow F^{B}\Rightarrow F^{Z}$
\item Set $A=B\Rightarrow Z$ and apply $\text{fmap}_{2}$ to the identity
$\text{id}^{\left(B\Rightarrow Z\right)\Rightarrow\left(B\Rightarrow Z\right)}$:
obtain $\text{ap}^{[B,Z]}:F^{B\Rightarrow Z}\Rightarrow F^{B}\Rightarrow F^{Z}\equiv\text{fmap}_{2}\left(\text{id}\right)$
\item The functions \texttt{\textcolor{blue}{\footnotesize{}fmap$_{2}$}}
and \texttt{\textcolor{blue}{\footnotesize{}ap}} are computationally
equivalent:{\footnotesize{}
\[
\text{fmap}_{2}\,f^{A\Rightarrow B\Rightarrow Z}=\text{fmap}\,f\circ\text{ap}
\]
\[
\xymatrix{\xyScaleY{0.2pc}\xyScaleX{3pc} & F^{B\Rightarrow Z}\ar[rd]\sp(0.45){\text{ap}}\\
F^{A}\ar[ru]\sp(0.45){\text{fmap}\,f}\ar[rr]\sb(0.45){\text{fmap}_{2}\,(f^{A\Rightarrow B\Rightarrow Z})} &  & \left(F^{B}\Rightarrow F^{Z}\right)
}
\]
}{\footnotesize \par}
\item The functions \texttt{\textcolor{blue}{\footnotesize{}fmap$_{3}$}},
\texttt{\textcolor{blue}{\footnotesize{}fmap$_{4}$}} etc.\ can be
defined similarly:{\footnotesize{}
\[
\text{fmap}_{3}\,f^{A\Rightarrow B\Rightarrow C\Rightarrow Z}=\text{fmap}\,f\circ\text{ap}\circ\text{fmap}_{F^{B}\Rightarrow?}\text{ap}
\]
\[
\xymatrix{\xyScaleY{0.2pc}\xyScaleX{3pc} & F^{B\Rightarrow C\Rightarrow Z}\ar[r]\sp(0.45){\text{ap}^{[B,C\Rightarrow Z]}} & \left(F^{B}\Rightarrow F^{C\Rightarrow Z}\right)\ar[rd]\sp(0.55){\text{fmap}_{F^{B}\Rightarrow?}\text{ap}^{[C,Z]}}\\
F^{A}\ar[ru]\sp(0.45){\text{fmap}\,f}\ar[rrr]\sb(0.45){\text{fmap}_{3}\,(f^{A\Rightarrow B\Rightarrow C\Rightarrow Z})} &  &  & \left(F^{B}\Rightarrow F^{C}\Rightarrow F^{Z}\right)
}
\]
}{\footnotesize \par}
\item Using the infix syntax will get rid of {\footnotesize{}$\text{fmap}_{F^{B}\Rightarrow?}\text{ap}$}
(see example code)
\begin{itemize}
\item Note the pattern: a natural transformation is equivalent to a lifting
\end{itemize}
\end{itemize}
\end{frame}

\begin{frame}{Deriving the \texttt{\textcolor{blue}{\footnotesize{}zip}} operation
from \texttt{\textcolor{blue}{\footnotesize{}map2}} }
\begin{itemize}
\item Note: Function types $A\Rightarrow B\Rightarrow C$ and $A\times B\Rightarrow C$
are equivalent
\item Uncurry $\text{fmap}_{2}$ to $\text{fmap2}:\left(A\times B\Rightarrow C\right)\Rightarrow F^{A}\times F^{B}\Rightarrow F^{C}$ 
\item Compute $\text{fmap2}\left(f\right)$ with $f=\text{id}^{A\times B\Rightarrow A\times B}$,
expecting to obtain a simpler natural transformation: 
\[
\text{zip}:F^{A}\times F^{B}\Rightarrow F^{A\times B}
\]
 
\item This is quite similar to \texttt{\textcolor{blue}{\footnotesize{}zip}}
for lists:

\texttt{\textcolor{blue}{\footnotesize{}List(1, 2).zip(List(10, 20))
= List((1, 10), (2, 20))}}{\footnotesize \par}
\item The functions \texttt{\textcolor{blue}{\footnotesize{}zip}} and \texttt{\textcolor{blue}{\footnotesize{}fmap2}}
are computationally equivalent:{\footnotesize{}
\begin{align*}
\text{zip} & =\text{fmap2}\left(\text{id}\right)\\
\text{fmap2}\,(f^{A\times B\Rightarrow C}) & =\text{zip}\circ\text{fmap}\,f
\end{align*}
\[
\xymatrix{\xyScaleY{0.2pc}\xyScaleX{3pc} & F^{A\times B}\ar[rd]\sp(0.65){\ \ \text{fmap}\,f^{A\times B\Rightarrow C}}\\
F^{A}\times F^{B}\ar[ru]\sp(0.5){\text{zip}}\ar[rr]\sb(0.6){\text{fmap2}\,(f^{A\times B\Rightarrow C})} &  & F^{C}
}
\]
}{\footnotesize \par}
\item The functor $F$\smallskip{}
}}{\footnotesize \par}

\texttt{\textcolor{blue}{\footnotesize{}}}%
\begin{minipage}[c][1\totalheight][t]{0.49\columnwidth}%
\begin{lyxcode}
\textcolor{blue}{\footnotesize{}for~\{~x~$\leftarrow$~pure(a)}{\footnotesize \par}

\textcolor{blue}{\footnotesize{}~~~~~~y~$\leftarrow$~cont}{\footnotesize \par}

\textcolor{blue}{\footnotesize{}\}~yield~g(x,~y)}{\footnotesize \par}
\end{lyxcode}
%
\end{minipage}\texttt{\textcolor{blue}{\footnotesize{}\hfill{}}}%
\begin{minipage}[c][1\totalheight][t]{0.49\columnwidth}%
\begin{lyxcode}
\textcolor{blue}{\footnotesize{}for~\{}{\footnotesize \par}

\textcolor{blue}{\footnotesize{}~~y~$\leftarrow$~cont}{\footnotesize \par}

\textcolor{blue}{\footnotesize{}\}~yield~g(a,~y)}{\footnotesize \par}
\end{lyxcode}
%
\end{minipage}\texttt{\textcolor{blue}{\footnotesize{}\hfill{}\medskip{}
}}{\footnotesize \par}

Write this in terms of \texttt{\textcolor{blue}{\footnotesize{}map2}}
to obtain the \textbf{identity laws} for \texttt{\textcolor{blue}{\footnotesize{}map2}}
and \texttt{\textcolor{blue}{\footnotesize{}pure}}:
\begin{lyxcode}
\vspace{-0.1cm}\textcolor{blue}{\footnotesize{}map2(pure(a),~cont)(g)~=~cont.map~\{~y~$\Rightarrow$~g(a,~y)~\}}~

\textcolor{blue}{\footnotesize{}map2(cont,~pure(b))(g)~=~cont.map~\{~x~$\Rightarrow$~g(x,~b)~\}}~
\end{lyxcode}
\end{frame}

\begin{frame}{Deriving the laws for \texttt{\textcolor{blue}{\footnotesize{}zip}}:
naturality}
\begin{itemize}
\item \vspace{-0.2cm}Rewrite the laws for \texttt{\textcolor{blue}{\footnotesize{}map2}}
in a short notation:{\footnotesize{}
\begin{align*}
\text{fmap2}\left(g^{A\times B\Rightarrow C}\right)\left(f^{\uparrow}q_{1}\times q_{2}\right) & =\text{fmap2}\left(\left(f\times\text{id}\right)\circ g\right)\left(q_{1}\times q_{2}\right)\\
\text{fmap2}\left(g^{A\times B\Rightarrow C}\right)\left(q_{1}\times f^{\uparrow}q_{2}\right) & =\text{fmap2}\left(\left(\text{id}\times f\right)\circ g\right)\left(q_{1}\times q_{2}\right)\\
\text{fmap2}\left(g_{1.23}\right)\left(q_{1}\times\text{fmap2}\left(\text{id}\right)\left(q_{2}\times q_{3}\right)\right) & =\text{fmap2}\left(g_{12.3}\right)\left(\text{fmap2}\left(\text{id}\right)\left(q_{1}\times q_{2}\right)\times q_{3}\right)\\
\text{fmap2}\left(g^{A\times B\Rightarrow C}\right)\left(\text{pure}\,a^{A}\times q_{2}^{F^{B}}\right) & =\left(b\Rightarrow g\left(a\times b\right)\right)^{\uparrow}q_{2}\\
\text{fmap2}\left(g^{A\times B\Rightarrow C}\right)\left(q_{1}^{F^{A}}\times\text{pure}\,b^{B}\right) & =\left(a\Rightarrow g\left(a\times b\right)\right)^{\uparrow}q_{1}
\end{align*}
}{\footnotesize \par}
\item Express \texttt{\textcolor{blue}{\footnotesize{}map2}} through \texttt{\textcolor{blue}{\footnotesize{}zip}}:{\footnotesize{}
\begin{align*}
\text{fmap}_{2}\,g^{A\times B\Rightarrow C}\left(q_{1}^{F^{A}}\times q_{2}^{F^{B}}\right) & \equiv\left(\text{zip}\circ g^{\uparrow}\right)\left(q_{1}\times q_{2}\right)\\
\text{fmap}_{2}\,g^{A\times B\Rightarrow C} & \equiv\text{zip}\circ g^{\uparrow}
\end{align*}
}{\footnotesize \par}
\item Combine the two naturality laws into one by using two functions $f_{1}$,
$f_{2}$:{\footnotesize{}
\begin{align*}
\left(f_{1}^{\uparrow}\times f_{2}^{\uparrow}\right)\circ\text{fmap2}\,g & =\text{fmap2}\left(\left(f_{1}\times f_{2}\right)\circ g\right)\\
\left(f_{1}^{\uparrow}\times f_{2}^{\uparrow}\right)\circ\text{zip}\circ g^{\uparrow} & =\text{zip}\circ\left(f_{1}\times f_{2}\right)^{\uparrow}\circ g^{\uparrow}
\end{align*}
}{\footnotesize \par}
\item \vspace{-0.2cm}The \textbf{naturality law} for \texttt{\textcolor{blue}{\footnotesize{}zip}}
then becomes: {\footnotesize{}$\left(f_{1}^{\uparrow}\times f_{2}^{\uparrow}\right)\circ\text{zip}=\text{zip}\circ\left(f_{1}\times f_{2}\right)^{\uparrow}$} 
\end{itemize}
\end{frame}

\begin{frame}{Deriving the laws for \texttt{\textcolor{blue}{\footnotesize{}zip}}:
associativity}
\begin{itemize}
\item Express \texttt{\textcolor{blue}{\footnotesize{}map2}} through \texttt{\textcolor{blue}{\footnotesize{}zip}}
and substitute into the associativity law:{\footnotesize{}
\[
g_{1.23}^{\uparrow}\left(\text{zip}\left(q_{1}\times\text{zip}\left(q_{2}\times q_{3}\right)\right)\right)=g_{12.3}^{\uparrow}\left(\text{zip}\left(\text{zip}\left(q_{1}\times q_{2}\right)\times q_{3}\right)\right)
\]
}{\footnotesize \par}
\item The arbitrary function $g$ is preceded by transformations of the
tuples,{\footnotesize{}
\[
a\times\left(b\times c\right)\equiv\left(a\times b\right)\times c\quad\text{(type isomorphism)}
\]
}{\footnotesize \par}
\item Assume that the isomorphism transformations are applied as needed:{\footnotesize{}
\[
\text{zip}\left(q_{1}\times\text{zip}\left(q_{2}\times q_{3}\right)\right)=\text{zip}\left(\text{zip}\left(q_{1}\times q_{2}\right)\times q_{3}\right)\quad\text{(\textbf{associativity law})}
\]
}{\footnotesize \par}
\item Identity laws seem to be complicated, e.g.\ the left identity:{\footnotesize{}
\[
g^{\uparrow}\left(\text{zip}\left(\text{pure}\,a\times q\right)\right)=\left(b\Rightarrow g\left(a\times b\right)\right)^{\uparrow}q
\]
}Replace \texttt{\textcolor{blue}{\footnotesize{}pure}} by a simpler
``wrapped unit'' method \texttt{\textcolor{blue}{\footnotesize{}unit:\ F{[}Unit{]}}}{\footnotesize{}
\[
\text{unit}^{F^{1}}\equiv\text{pure}\left(1\right);\quad\text{pure}(a^{A})=\left(1\Rightarrow a\right)^{\uparrow}\text{unit}
\]
}Then the left identity law can be simplified using left naturality:{\footnotesize{}
\[
g^{\uparrow}\left(\text{zip}\left(\left(\left(1\Rightarrow a\right)^{\uparrow}\text{unit}\right)\times q\right)\right)=\left(b\Rightarrow g\left(a\times b\right)\right)^{\uparrow}q
\]
}{\footnotesize \par}
\end{itemize}
\end{frame}

\begin{frame}{Constructions of applicative functors}
\begin{itemize}
\item Express 
\end{itemize}
\end{frame}

\begin{frame}{All non-parameterized exp-poly types are monoids}
\begin{itemize}
\item Express 
\end{itemize}
\end{frame}

\begin{frame}{All non-parameterized polynomial functors are applicative}
\begin{itemize}
\item Express 
\end{itemize}
\end{frame}

\begin{frame}{Definition and constructions of applicative contrafunctors}
\begin{itemize}
\item Express 
\end{itemize}
\end{frame}

\begin{frame}{All non-parameterized exp-poly contrafunctors are applicative}
\begin{itemize}
\item Express 
\end{itemize}
\end{frame}

\begin{frame}{Definition and constructions of applicative profunctors}
\begin{itemize}
\item Express 
\end{itemize}
\end{frame}

\begin{frame}{Exercises}
\begin{enumerate}
\item Show that $F^{A}\equiv\left(Z\Rightarrow A\right)\Rightarrow1+A$
is not applicative.
\end{enumerate}
\end{frame}

\end{document}
