\batchmode
\makeatletter
\def\input@path{{/Users/sergei.winitzki/Code/talks/ftt-fp/}}
\makeatother
\documentclass[english]{beamer}
\usepackage[T1]{fontenc}
\usepackage[latin9]{inputenc}
\setcounter{secnumdepth}{3}
\setcounter{tocdepth}{3}
\usepackage{babel}
\usepackage{amstext}
\ifx\hypersetup\undefined
  \AtBeginDocument{%
    \hypersetup{unicode=true,pdfusetitle,
 bookmarks=true,bookmarksnumbered=false,bookmarksopen=false,
 breaklinks=false,pdfborder={0 0 1},backref=false,colorlinks=true}
  }
\else
  \hypersetup{unicode=true,pdfusetitle,
 bookmarks=true,bookmarksnumbered=false,bookmarksopen=false,
 breaklinks=false,pdfborder={0 0 1},backref=false,colorlinks=true}
\fi

\makeatletter
%%%%%%%%%%%%%%%%%%%%%%%%%%%%%% Textclass specific LaTeX commands.
 % this default might be overridden by plain title style
 \newcommand\makebeamertitle{\frame{\maketitle}}%
 % (ERT) argument for the TOC
 \AtBeginDocument{%
   \let\origtableofcontents=\tableofcontents
   \def\tableofcontents{\@ifnextchar[{\origtableofcontents}{\gobbletableofcontents}}
   \def\gobbletableofcontents#1{\origtableofcontents}
 }
 \newenvironment{lyxcode}
   {\par\begin{list}{}{
     \setlength{\rightmargin}{\leftmargin}
     \setlength{\listparindent}{0pt}% needed for AMS classes
     \raggedright
     \setlength{\itemsep}{0pt}
     \setlength{\parsep}{0pt}
     \normalfont\ttfamily}%
    \def\{{\char`\{}
    \def\}{\char`\}}
    \def\textasciitilde{\char`\~}
    \item[]}
   {\end{list}}

%%%%%%%%%%%%%%%%%%%%%%%%%%%%%% User specified LaTeX commands.
\usetheme[secheader]{Boadilla}
\usecolortheme{seahorse}
\title[Chapter 7: Functor-lifted computations II]{Chapter 7: Computations lifted to a functor context II. Monads and semimonads}
\subtitle{Part 1: Intuitions, examples, use cases}
\author{Sergei Winitzki}
\date{2018-03-25}
\institute[ABTB]{Academy by the Bay}
\setbeamertemplate{headline}{} % disable headline at top
\setbeamertemplate{navigation symbols}{} % disable navigation bar at bottom
\usepackage[all]{xy}
\usepackage[nocenter]{qtree}
\makeatletter
% Macros to assist LyX with XYpic when using scaling.
\newcommand{\xyScaleX}[1]{%
\makeatletter
\xydef@\xymatrixcolsep@{#1}
\makeatother
} % end of \xyScaleX
\makeatletter
\newcommand{\xyScaleY}[1]{%
\makeatletter
\xydef@\xymatrixrowsep@{#1}
\makeatother
} % end of \xyScaleY

\makeatother

\begin{document}
\frame{\titlepage}
\begin{frame}{Computations within a functor context: Semimonads}


\framesubtitle{Intuitions behind adding more ``generator arrows''}

Example of nested iterations: {\footnotesize{}
\[
\sum_{i=1}^{n}\sum_{j=1}^{n}\sum_{k=1}^{n}f(i,j,k)
\]
}{\footnotesize \par}

Using Scala's \texttt{\textcolor{blue}{\footnotesize{}for}}/\texttt{\textcolor{blue}{\footnotesize{}yield}}
syntax (``functor block'')

\texttt{\textcolor{blue}{\footnotesize{}}}%
\begin{minipage}[t]{0.49\columnwidth}%
\begin{lyxcode}
\textcolor{blue}{\footnotesize{}(for~\{~i~$\leftarrow$~1~to~n}{\footnotesize \par}

\textcolor{blue}{\footnotesize{}~~~~j~$\leftarrow$~1~to~n}{\footnotesize \par}

\textcolor{blue}{\footnotesize{}~~~~k~$\leftarrow$~1~to~n}{\footnotesize \par}

\textcolor{blue}{\footnotesize{}~~\}~yield~f(i,~j,~k)}{\footnotesize \par}

\textcolor{blue}{\footnotesize{}).sum}{\footnotesize \par}
\end{lyxcode}
%
\end{minipage}\texttt{\textcolor{blue}{\footnotesize{}}}%
\begin{minipage}[t]{0.49\columnwidth}%
\begin{lyxcode}
\textcolor{blue}{\footnotesize{}(1~to~n).flatMap~\{~i~$\Rightarrow$}{\footnotesize \par}

\textcolor{blue}{\footnotesize{}~~~(1~to~n).flatMap~\{~j~$\Rightarrow$}{\footnotesize \par}

\textcolor{blue}{\footnotesize{}~~~~~(1~to~n).map~\{~k~$\Rightarrow$}{\footnotesize \par}

\textcolor{blue}{\footnotesize{}~~~~~~~f(i,~j,~k)}{\footnotesize \par}

\textcolor{blue}{\footnotesize{}~~\}\}\}.sum}{\footnotesize \par}
\end{lyxcode}
%
\end{minipage}\texttt{\textcolor{blue}{\footnotesize{}\medskip{}
}}{\footnotesize \par}
\begin{itemize}
\item \texttt{\textcolor{blue}{\footnotesize{}map}} replaces the last left
arrow, \texttt{\textcolor{blue}{\footnotesize{}flatMap}} replaces
other left arrows
\begin{itemize}
\item When the functor is \emph{also} filterable, we can use ``\texttt{\textcolor{blue}{\footnotesize{}if}}''
as well
\end{itemize}
\item Standard library defines \texttt{\textcolor{blue}{\footnotesize{}flatMap()}}
as replacement of \texttt{\textcolor{blue}{\footnotesize{}map() $\circ$
flatten}} 
\begin{itemize}
\item \texttt{\textcolor{blue}{\footnotesize{}(1 to n).map(j $\Rightarrow$
...).flatten}} is \texttt{\textcolor{blue}{\footnotesize{}(1 to n).flatMap(j
$\Rightarrow$ ...)}} 
\end{itemize}
\item Functors having \texttt{\textcolor{blue}{\footnotesize{}flatMap}}/\texttt{\textcolor{blue}{\footnotesize{}flatten}}
are ``flattenable'' or \textbf{semimonads}
\begin{itemize}
\item Most of them also have method \texttt{\textcolor{blue}{\footnotesize{}pure:\ A
$\Rightarrow$ F{[}A{]}}} and so are \textbf{monads}
\begin{itemize}
\item The method \texttt{\textcolor{blue}{\footnotesize{}pure}} is not relevant
in the functor block
\item We will not need \texttt{\textcolor{blue}{\footnotesize{}pure}} in
this part of the tutorial; focus on semimonads
\end{itemize}
\end{itemize}
\end{itemize}
\end{frame}

\begin{frame}{How \texttt{\textcolor{blue}{\footnotesize{}flatMap}} works with lists}

\begin{itemize}
\item consider \texttt{\textcolor{blue}{\footnotesize{}List(x1, x2, x3).flatMap(x
$\Rightarrow$ f(x))}} 
\item assume that 
\end{itemize}
\begin{lyxcode}
\textcolor{blue}{\footnotesize{}f:~X~$\Rightarrow$~List{[}Y{]}}{\footnotesize \par}

\textcolor{blue}{\footnotesize{}f(x1)~=~List(y0,~y1)}{\footnotesize \par}

\textcolor{blue}{\footnotesize{}f(x2)~=~List(y2)}{\footnotesize \par}

\textcolor{blue}{\footnotesize{}f(x3)~=~List(y3,~y4,~y5,~y6)}{\footnotesize \par}
\end{lyxcode}
\begin{itemize}
\item then the result is \texttt{\textcolor{blue}{\footnotesize{}List(y0,
y1, y2, y3, y4, y5, y6)}} 
\item if we first do \texttt{\textcolor{blue}{\footnotesize{}.map(f)}} then
\texttt{\textcolor{blue}{\footnotesize{}flatten}}:
\end{itemize}
\begin{lyxcode}
\textcolor{blue}{\footnotesize{}List(x1,~x2,~x3).map(f).flatten~=}{\footnotesize \par}

\textcolor{blue}{\footnotesize{}~~List(List(y0,~y1),~List(y2),~List(y3,~y4,~y5,~y6)).flatten~=}{\footnotesize \par}

\textcolor{blue}{\footnotesize{}~~List(y0,~y1,~y2,~y3,~y4,~y5,~y6)}~
\end{lyxcode}
\end{frame}

\begin{frame}{What is \texttt{\textcolor{blue}{\footnotesize{}flatMap}} doing with
the data in a collection?}

Consider this schematic code, using \texttt{\textcolor{blue}{\footnotesize{}Seq}}
as the container type:\texttt{\textcolor{blue}{\footnotesize{} }}%
\begin{minipage}[t]{0.49\columnwidth}%
\begin{lyxcode}
\textcolor{blue}{\footnotesize{}val~result~=~for~\{}{\footnotesize \par}

\textcolor{blue}{\footnotesize{}~~i~$\leftarrow$~1~to~m}{\footnotesize \par}

\textcolor{blue}{\footnotesize{}~~j~$\leftarrow$~1~to~n}{\footnotesize \par}

\textcolor{blue}{\footnotesize{}~~x~=~f(i,~j)}{\footnotesize \par}

\textcolor{blue}{\footnotesize{}~~k~$\leftarrow$~1~to~p}{\footnotesize \par}

\textcolor{blue}{\footnotesize{}~~y~=~g(i,~j,~k)}{\footnotesize \par}

\textcolor{blue}{\footnotesize{}\}~yield~h(x,y)}{\footnotesize \par}
\end{lyxcode}
%
\end{minipage}\texttt{\textcolor{blue}{\footnotesize{}}}%
\begin{minipage}[t]{0.49\columnwidth}%
\begin{lyxcode}
\textcolor{blue}{\footnotesize{}val~result~=~\{}{\footnotesize \par}

\textcolor{blue}{\footnotesize{}~~(1~to~m).flatMap~\{~i~$\Rightarrow$}{\footnotesize \par}

\textcolor{blue}{\footnotesize{}~~~~(1~to~n).flatMap~\{~j~$\Rightarrow$

\framesubtitle{Motivation for the choice of the type constructors $\text{Writer}^{A}$,
$\text{Reader}^{A}$, $\text{State}^{A}$, $\text{Cont}^{A}$}

We want previous values to be transformed via \texttt{\textcolor{blue}{\footnotesize{}flatMap}}
to next values
\begin{itemize}
\item \texttt{\textcolor{blue}{\footnotesize{}Writer}}: a computation $\left(A\Rightarrow B\right)$
and some info ($W$) about it
\begin{itemize}
\item $x^{A}\Rightarrow f(x):B$ and $x^{A}\Rightarrow g(x):W$; the type
is $\left(A\Rightarrow B\right)\times\left(A\Rightarrow W\right)$
\item this function should have type $A\Rightarrow\text{Writer}^{B}$, hence
$\text{Writer}^{B}\equiv B\times W$ 
\begin{itemize}
\item use the ``arithmetic'' Curry-Howard to transform types: $b^{a}w^{a}=(bw)^{a}$
\end{itemize}
\end{itemize}
\item \texttt{\textcolor{blue}{\footnotesize{}Reader}}: Read-only context,
or ``environment'' of type $E$
\begin{itemize}
\item $x^{A}\Rightarrow f(r,x):B$ where $r^{E}$ is fixed; the type is
$A\times E\Rightarrow B$
\item this function should have type $A\Rightarrow\text{Reader}^{B}$, hence
$\text{Reader}^{B}\equiv E\Rightarrow B$
\begin{itemize}
\item we used the ``arithmetic'' Curry-Howard to transform $b^{ae}=(b^{e})^{a}$
\end{itemize}
\end{itemize}
\item \texttt{\textcolor{blue}{\footnotesize{}Cont}}: A computation that
registers an asynchronous callback
\begin{itemize}
\item $x^{A}\Rightarrow f(cb):1$ where $cb:B\Rightarrow1$ (usually, callbacks
return \texttt{\textcolor{blue}{\footnotesize{}Unit}})
\item the type is{\footnotesize{} $A\Rightarrow\left(B\Rightarrow1\right)\Rightarrow1$};
this function should have type {\footnotesize{}$A\Rightarrow\text{Cont}^{B}$},
hence{\footnotesize{} $\text{Cont}^{B}\equiv\left(B\Rightarrow1\right)\Rightarrow1$}{\footnotesize \par}
\item generalize to {\footnotesize{}$\text{Cont}^{A}\equiv\left(A\Rightarrow R\right)\Rightarrow R$
}where $R$ is a fixed ``result'' type
\end{itemize}
\item \texttt{\textcolor{blue}{\footnotesize{}State}}: A computation can
update state ($S$) while producing a result
\begin{itemize}
\item $x^{A}\Rightarrow f(x,s)$ and $s^{S}:=g(x,s)$; the type is{\footnotesize{}
$\left(A\times S\Rightarrow B\right)\times\left(A\times S\Rightarrow S\right)$}{\footnotesize \par}
\item this will be $A\Rightarrow\text{State}^{B}$ if {\footnotesize{}$\text{State}^{B}\equiv\left(S\Rightarrow B\right)\times\left(S\Rightarrow S\right)\equiv S\Rightarrow B\times S$ }{\footnotesize \par}
\begin{itemize}
\item we used the ``arithmetic'' Curry-Howard: $b^{as}s^{as}=(b^{s}s^{s})^{a}=\left(\left(bs\right)^{s}\right)^{a}$
\end{itemize}
\end{itemize}
\end{itemize}
\end{frame}

\begin{frame}{Exercises I}
\begin{enumerate}
\item For a given \texttt{\textcolor{blue}{\footnotesize{}Set{[}Int{]}}},
compute all subsets $\left(w,x,y,z\right)$ of size 4 such that $w<x<y<z$
and $w+x=y+z$
\item Given 3 sequences $xs$, $ys$, $zs$ of type \texttt{\textcolor{blue}{\footnotesize{}Seq{[}Int{]}}},
compute all $\left(x,y,z\right)$ such that $x\in xs$, $y\in ys$,
$z\in zs$ and $x<y<z$ and $x+y+z<10$
\item How many chess queens can avoid each other on an $3\times3\times3$
cube?
\item Write a tiny library for arithmetic using \texttt{\textcolor{blue}{\footnotesize{}Future}}'s;
use it to compute $1+2+...+100$ via \texttt{\textcolor{blue}{\footnotesize{}for}}/\texttt{\textcolor{blue}{\footnotesize{}yield}}
and verify the result. E.g.\ implement: 
\begin{lyxcode}
\textcolor{blue}{\footnotesize{}const:~Int~$\Rightarrow$~Future{[}Int{]}}{\footnotesize \par}

\textcolor{blue}{\footnotesize{}add(x:~Int):~Int~$\Rightarrow$~Future{[}Int{]}}{\footnotesize \par}

\textcolor{blue}{\footnotesize{}isEqual(x:~Int):~Int~$\Rightarrow$~Future{[}Boolean{]}~}{\footnotesize \par}
\end{lyxcode}
\item Read a file into a string and write it to another file using Java
\texttt{\textcolor{blue}{\footnotesize{}Files}} and \texttt{\textcolor{blue}{\footnotesize{}Paths}}
API\texttt{\textcolor{blue}{\footnotesize{}. }}Use \texttt{\textcolor{blue}{\footnotesize{}Try}}
and \texttt{\textcolor{blue}{\footnotesize{}for}}/\texttt{\textcolor{blue}{\footnotesize{}yield}}
to make this safe.
\item Given a semigroup $W$, make a semimonad out of $F^{A}\equiv E\Rightarrow A\times W$ 
\item Implement a semimonad instance for the (recursive) type constructor
$F^{A}=A+A\times A+F^{A}+F^{A}\times F^{A}$
\item Find the largest prime number below 1000 via a simple \href{https://en.wikipedia.org/wiki/Sieve_of_Eratosthenes}{Sieve of Eratosthenes};
use the \texttt{\textcolor{blue}{\footnotesize{}State{[}S, Int{]}}}
monad with \texttt{\textcolor{blue}{\footnotesize{}S = Array{[}Boolean{]}}} 
\end{enumerate}
\end{frame}

\end{document}
