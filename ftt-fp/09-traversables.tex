\batchmode
\makeatletter
\def\input@path{{/Users/sergei.winitzki/Code/talks/ftt-fp/}}
\makeatother
\documentclass[english]{beamer}
\usepackage[T1]{fontenc}
\usepackage[latin9]{inputenc}
\setcounter{secnumdepth}{3}
\setcounter{tocdepth}{3}
\usepackage{babel}
\usepackage{amstext}
\usepackage[all]{xy}
\ifx\hypersetup\undefined
  \AtBeginDocument{%
    \hypersetup{unicode=true,pdfusetitle,
 bookmarks=true,bookmarksnumbered=false,bookmarksopen=false,
 breaklinks=false,pdfborder={0 0 1},backref=false,colorlinks=true}
  }
\else
  \hypersetup{unicode=true,pdfusetitle,
 bookmarks=true,bookmarksnumbered=false,bookmarksopen=false,
 breaklinks=false,pdfborder={0 0 1},backref=false,colorlinks=true}
\fi

\makeatletter
%%%%%%%%%%%%%%%%%%%%%%%%%%%%%% Textclass specific LaTeX commands.
 % this default might be overridden by plain title style
 \newcommand\makebeamertitle{\frame{\maketitle}}%
 % (ERT) argument for the TOC
 \AtBeginDocument{%
   \let\origtableofcontents=\tableofcontents
   \def\tableofcontents{\@ifnextchar[{\origtableofcontents}{\gobbletableofcontents}}
   \def\gobbletableofcontents#1{\origtableofcontents}
 }

%%%%%%%%%%%%%%%%%%%%%%%%%%%%%% User specified LaTeX commands.
\usetheme[secheader]{Boadilla}
\usecolortheme{seahorse}
\title[Chapter 9: Traversable (contra)functors]{Chapter 9: Traversable functors and contrafunctors}
%\subtitle{Part 2: Their laws and structure}
\author{Sergei Winitzki}
\date{2018-08-08}
\institute[ABTB]{Academy by the Bay}
\setbeamertemplate{headline}{} % disable headline at top
\setbeamertemplate{navigation symbols}{} % disable navigation bar at bottom
\usepackage[all]{xy}
\makeatletter
% Macros to assist LyX with XYpic when using scaling.
\newcommand{\xyScaleX}[1]{%
\makeatletter
\xydef@\xymatrixcolsep@{#1}
\makeatother
} % end of \xyScaleX
\makeatletter
\newcommand{\xyScaleY}[1]{%
\makeatletter
\xydef@\xymatrixrowsep@{#1}
\makeatother
} % end of \xyScaleY

\makeatother

\begin{document}
\frame{\titlepage}
\begin{frame}{Motivation for the \texttt{\textcolor{blue}{\footnotesize{}traverse}}
operation}
\begin{itemize}
\item Consider data of type $\text{List}^{A}$ and processing $f:A\Rightarrow\text{Future}^{B}$
\item Typically, we want to wait until the entire data set is processed
\item What we need is $\text{List}^{A}\Rightarrow\left(A\Rightarrow\text{Future}^{B}\right)\Rightarrow\text{Future}^{\text{List}^{B}}$
\item Generalize: $L^{A}\Rightarrow\left(A\Rightarrow F^{B}\right)\Rightarrow F^{L^{B}}$
for some type constructors $F$, $L$
\item This operation is called \texttt{\textcolor{blue}{\footnotesize{}traverse}} 
\begin{itemize}
\item How to implement it: for example, a 3-element list is $A\times A\times A$
\item Consider $L^{A}\equiv A\times A\times A$, apply $\text{map}\,f$
and get $F^{B}\times F^{B}\times F^{B}$
\item We will get $F^{L^{B}}\equiv F^{B\times B\times B}$ if we can apply
\texttt{\textcolor{blue}{\footnotesize{}zip}} as $F^{B}\times F^{B}\Rightarrow F^{B\times B}$
\end{itemize}
\item So we need to assume that $F$ is applicative
\item In Scala, we have \texttt{\textcolor{blue}{\footnotesize{}Future.traverse()}}
that assumes $L$ to be a sequence
\begin{itemize}
\item This is the iconic example that fixes the requirements
\end{itemize}
\item Questions:
\begin{itemize}
\item Which functors $L$ can have this operation?
\item Can we express \texttt{\textcolor{blue}{\footnotesize{}traverse}}
through a simpler operation?
\item What are the required laws for \texttt{\textcolor{blue}{\footnotesize{}traverse}}?
\item What about contrafunctors or profunctors?
\end{itemize}
\end{itemize}
\end{frame}

\begin{frame}{Deriving the \texttt{\textcolor{blue}{\footnotesize{}sequence}} operation}
\begin{itemize}
\item \vspace{-0.1cm}The type signature of \texttt{\textcolor{blue}{\footnotesize{}traverse}}
is a complicated ``lifting''
\begin{itemize}
\item A ``lifting'' is always equivalent to a simpler natural transformation
\end{itemize}
\item To derive it, ask: what is missing from \texttt{\textcolor{blue}{\footnotesize{}fmap}}
to do the job of \texttt{\textcolor{blue}{\footnotesize{}traverse}}?{\footnotesize{}
\[
\text{fmap}:(A\Rightarrow F^{B})\Rightarrow L^{A}\Rightarrow L^{F^{B}}
\]
}{\footnotesize \par}
\item We need $F^{L^{B}}$, but the \texttt{\textcolor{blue}{\footnotesize{}traverse}}
operation gives us $L^{F^{B}}$ instead
\begin{itemize}
\item What's missing is a natural transformation \texttt{\textcolor{blue}{\footnotesize{}sequence}}
$:L^{F^{B}}\Rightarrow F^{L^{B}}$ 
\end{itemize}
\item The functions \texttt{\textcolor{blue}{\footnotesize{}traverse}} and
\texttt{\textcolor{blue}{\footnotesize{}sequence}} are computationally
equivalent:{\footnotesize{}
\[
\text{trav}\,f^{\underline{A\Rightarrow F^{B}}}=\text{fmap}\,f\circ\text{seq}
\]
\[
\xymatrix{\xyScaleY{0.2pc}\xyScaleX{3pc} & L^{F^{B}}\ar[rd]\sp(0.45){\text{seq}}\\
L^{A}\ar[ru]\sp(0.45){\text{fmap}\,f}\ar[rr]\sb(0.45){\text{trav}\,(f^{\underline{A\Rightarrow F^{B}}})} &  & F^{L^{B}}
}
\]
}Here $F$ is an arbitrary applicative functor
\begin{itemize}
\item Keep in mind the example \texttt{\textcolor{blue}{\footnotesize{}Future.sequence}}
$:\text{List}^{\text{Future}^{X}}\Rightarrow\text{Future}^{\text{List}^{X}}$
\item Examples: $\text{List}$, all ``finite'' polynomial functors (see
\href{http://www.cs.ox.ac.uk/jeremy.gibbons/publications/uitbaf.pdf}{Bird et al., 2013})
\item Non-traversable: $L^{A}\equiv R\Rightarrow A$; lazy lists (``infinite
streams'')
\begin{itemize}
\item Note: We \emph{cannot have} the opposite transformation $F^{L^{B}}\Rightarrow L^{F^{B}}$
\end{itemize}
\end{itemize}
\end{itemize}
\end{frame}

\begin{frame}{Motivation for the laws of the \texttt{\textcolor{blue}{\footnotesize{}traverse}}
operation }
\begin{itemize}
\item \vspace{-0.15cm}The \href{https://arxiv.org/pdf/1202.2919.pdf}{\textquotedblleft law of traversals\textquotedblright{} paper}
(2012) argues that \texttt{\textcolor{blue}{\footnotesize{}traverse}}
should ``visit each element'' of the container $L^{A}$ exactly
once, and evaluate each corresponding ``effect'' $F^{B}$ exactly
once; then they formulate the laws
\item To derive the laws, use the ``lifting'' intuition for \texttt{\textcolor{blue}{\footnotesize{}traverse}},{\footnotesize{}
\[
\text{trav}:(A\Rightarrow F^{B})\Rightarrow L^{A}\Rightarrow F^{L^{B}}
\]
}{\footnotesize \par}
\end{itemize}
{\footnotesize{}L}ook for ``identity'' and ``composition''``identity'' and ``composition'' laws:
\begin{enumerate}
\item ``Identity'' as \texttt{\textcolor{blue}{\footnotesize{}pure}} $:A\Rightarrow F^{A}$
must be lifted to $\text{pure}:L^{A}\Rightarrow F^{L^{A}}$
\item ``Identity'' as $\text{id}^{\underline{A\Rightarrow A}}$ with $F^{A}\equiv A$
(identity functor) lifted to $\text{id}^{\underline{L^{A}\Rightarrow L^{A}}}$
\item ``Compose'' $f:A\Rightarrow F^{B}$ and $g:B\Rightarrow G^{C}$
to get $h:A\Rightarrow F^{G^{C}}$, where $F$, $G$ are applicative;
a traversal with $h$ maps $L^{A}$ to $F^{G^{L^{C}}}$ and must be
somehow equal to the composition of traversals with $f$ and then
with $g$
\end{enumerate}
Questions:
\begin{itemize}
\item Are the laws for the \texttt{\textcolor{blue}{\footnotesize{}sequence}}
operation simpler?
\item Are all these laws independent?
\item What functors $L$ satisfy these laws \emph{for all} applicative functors
$F$?
\end{itemize}
\end{frame}

\begin{frame}{Formulation of the laws for \texttt{\textcolor{blue}{\footnotesize{}traverse}} }

\vspace{-0.15cm}The 
\end{frame}

\begin{frame}{Derivation of the laws for \texttt{\textcolor{blue}{\footnotesize{}sequence}} }

\vspace{-0.15cm}The
\end{frame}

\begin{frame}{Constructions of traversable functors}

\vspace{-0.15cm}The 
\end{frame}

\begin{frame}{Traversable profunctors}

\vspace{-0.15cm}The
\end{frame}

\begin{frame}{Traversability with respect to profunctors}

\vspace{-0.15cm}The
\end{frame}

\begin{frame}{Foldable functors}

\vspace{-0.15cm}The 
\end{frame}

\begin{frame}{Exercises}
\begin{enumerate}
\item {\footnotesize{}\vspace{-0.15cm}Show that any traversable functor
$L$ admits a method 
\[
\text{consume}:(L^{A}\Rightarrow B)\Rightarrow L^{F^{A}}\Rightarrow F^{B}
\]
for any applicative functor $F$. Show that }\texttt{\textcolor{blue}{\footnotesize{}traverse}}{\footnotesize{}
and }\texttt{\textcolor{blue}{\footnotesize{}consume}}{\footnotesize{}
are equivalent.}{\footnotesize \par}
\end{enumerate}
\end{frame}

\end{document}
