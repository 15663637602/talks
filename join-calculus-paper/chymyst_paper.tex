%% LyX 2.2.0 created this file.  For more info, see http://www.lyx.org/.
%% Do not edit unless you really know what you are doing.
\documentclass[english]{article}
\usepackage[T1]{fontenc}
\usepackage[latin9]{inputenc}
\usepackage{babel}
\usepackage{verbatim}
\usepackage{amsmath}
\usepackage[numbers]{natbib}
\usepackage[unicode=true]
 {hyperref}
\begin{document}

\title{Industry-strength join calculus: Declarative concurrent programming
with \texttt{Chymyst}}

\author{Sergei Winitzki}
\maketitle
\begin{abstract}
Join calculus (JC) is a declarative message-passing concurrency formalism
that has been ignored by the software engineering community, despite
its significant promise as a means of solving the problems of concurrent
programming. I introduce \texttt{Chymyst}, a new open-source framework
that aims to bring industry-strength JC programming to Scala practitioners.
Taking advantage of its embedding into the Scala language, \texttt{Chymyst}
enhances JC with features such as arbitrary non-linear join patterns
with guard conditions, synchronous rendezvous, time-outs, and incremental
construction of join definitions. The current implementation also
performs static analysis of user code, early error detection, and
automatic performance optimizations. To ease the learning curve for
engineers unfamiliar with the concepts of JC, I develop a pedagogical
presentation of JC as an evolution of the well-known Actor model whereby
actors are made type-safe, immutable, and are automatically managed.
After a comparison of \texttt{Chymyst} with the popular \texttt{Akka}
library, I identify a comprehensive set of additional features necessary
to make JC an industry-ready concurrency paradigm. These features
include APIs for unit testing, performance monitoring, and fault tolerance,
and are next steps on the \texttt{Chymyst} project's development roadmap.
\end{abstract}

\section{Introduction and summary}

Advanced programming models developed by the theoretical computer
science community are often ignored by software practitioners. One
such case is join calculus (JC)~\citep{FouGon1996}, which can be
seen as a DSL (domain-specific language) for high-level, declarative,
functional concurrent programming. Given the high importance of concurrent
programming and a growing adoption of functional languages, one would
expect that software practitioners would take advantage of this high-level
and type-safe concurrency paradigm. Nevertheless, there appears to
be no practical adoption of JC by the software industry.\footnote{A Google search yields several academic projects but no mentions of
industrial JC use.} Perhaps not coincidentally, there are very few open-source implementations
of JC available for download and use. The only fully maintained implementation
of JC is the JoCaml language~\citep{FouEA2003}.

A significant barrier for industry adoption is the lack of suitable
documentation and example code. The existing documentation and tutorials
for JC, such as the JoCaml user's manual\footnote{ See \href{http://jocaml.inria.fr/doc/index.html}{jocaml.inria.fr/doc/index.html}.},
the original authors' introduction to JC~\citet{FouGon2000}, and
the lecture notes~\citep{FouEA2003} were intended for graduate students
in computer science and are largely incomprehensible to software engineers.
Effective JC programming requires a certain paradigm shift and facility
with JC-specific design patterns, which is not readily achieved without
working through numerous examples.

In this paper, I present a new open-source implementation of JC as
a library called \texttt{Chymyst},\footnote{The name is borrowed from the early treatise~\citep{Boy1661} by
Robert Boyle, who was one of the founders of the science of chemistry.} providing an embedded Scala DSL and a light-weight runtime. The main
design focus of \texttt{Chymyst} is to enable high-level, declarative
concurrency in idiomatic Scala, using the JC paradigm. At the same
time, I aim to provide industry-strength features that are typically
not considered by academic presentations of JC, such as performance
tuning, fault tolerance, or unit testing APIs. Finally, the \texttt{Chymyst}
project offers tutorial documentation adapted to the software developer
audience. In these ways, I hope to enable industry adoption of this
promising concurrency paradigm.

\subsection{Contributions of this paper}

I describe the main design decisions made in the \texttt{Chymyst}
project while implementing join calculus as an embedded DSL in Scala. 

\texttt{Chymyst} lifts several restrictions that are present in most
other JC projects, and offers some additional features:
\begin{itemize}
\item separate definition of channel names and processes
\item arbitrary non-linear patterns and guards in process definitions
\item synchronous rendezvous
\item first-class processes with arbitrary process bodies
\item incremental construction of join definitions from processes
\item automatic performance optimizations
\item static code analysis and early error detection
\end{itemize}
I argue that certain new facilities need to be added to a JC implementation
in order to make it viable for industry adoption. These facilities
include:
\begin{itemize}
\item explicit thread pools for performance tuning
\item time-outs for synchronous channels
\item interoperability with \texttt{Future}s
\item APIs for unit testing and debugging
\item per-process fault tolerance settings
\item message pipelining
\end{itemize}
I describe the current implementation of these facilities in \texttt{Chymyst}.

To better explain the concepts of JC to new software developers, I
avoid the traditional academic terminology (message / channel / process
/ join definition). Instead, I show how to describe JC as an evolution
of the Actor model, which is ideal for developers already familiar
with the Akka library. When introducing JC from scratch, I rely on
the ``abstract chemical machine'' metaphor and use the corresponding
terminology (molecule / emitter / reaction / reaction site), which
is more visual and intuitive. Synchronous channels (``blocking emitters'')
are most easily understood by carrying out a continuation-passing
code transformation from blocking code to code that uses only asynchronous,
non-blocking channels. The syntax used by \texttt{Chymyst} for blocking
channels makes this code transformation more transparent.

\subsection{Previous work}

Since its invention more than 20 years ago, join calculus has been
implemented by a number of academic researchers, typically by creating
an entirely new JC-based programming language or by patching an existing
language. It is hard to assess the scope and practical use of these
implementations, since most of them appear to be proof-of-concept
projects developed to accompany academic publications. 

Here I will not attempt to survey the theoretical advances made by
those researchers. Since the main goal of the \texttt{Chymyst} project
is to enable industry acceptance of JC, I will focus on the practical
availability and usability of the existing JC implementations.

JoCaml was one of the first implementations of JC~\citep{FouEA2003},
and remains today the best-supported one. This implementation is a
patch for the OCaml compiler, which is fully compatible with the OCaml
library ecosystem. 

M.~Odersky created a new language called ``Funnel'', based on the
JC paradigm~\citep{Ode2000}. The Funnel project appears to be abandoned,
since M.~Odersky went on to develop Scala, which does not include
any concepts or features of JC.

G.~S.~von Itzstein implemented JC as a patch for the early Java
compiler~\citep{Von2004}. The ``Join Java'' project appears to
be abandoned.

The first appearance of JC in Scala was a \textquotedblleft Join in
Scala\textquotedblright{} compiler patch by V.~Cremet (2003).\footnote{See \href{http://lampwww.epfl.ch/~cremet/misc/join_in_scala/index.html}{lampwww.epfl.ch/$\sim$cremet/misc/join\_{}in\_{}scala/}.}
The syntax of Scala has changed radically since 2003, rendering the
project unusable.

``Joinads'' is a set of compiler patches for F\# and Haskell, developed
by T.~Petricek~\citep{PetSym2011}. The project is not maintained.

T.~Rompf implemented an experimental (unnamed) language based on
JC and illustrated its use for important application design patterns,
such as ``fork/join'' synchronization or asynchronous continuations~\citep{Rom2007}.
The project appears to be abandoned, as T.~Rompf moved on to research
on multi-stage compilation~\citep{Rom2012}.

Creating a \emph{new} programming language, either from scratch or
via compiler patches, has been a common pattern in JC implementations.
The reason seems to be the difficulty of accommodating join definitions
within the syntax of existing languages. Short-lived projects such
as Polyphonic C\#~\citep{BenFou2002}, C$\omega$~\citep{Rus2007},
Join Diesel~\citep{Ose2005}, JErlang~\citep{PloEis2009}, and Hume~\citep{HamEA2006}
also follow that pattern. All these new languages have since been
abandoned by their creators. It appears that maintaining and supporting
a completely new research language for JC is hardly possible, even
for a corporation such as Microsoft. Therefore, we turn our attention
to implementations of JC as an embedded DSL in a well-established
programming language. 

C.~Russo created the ``Scalable Joins'' library for the .NET platform~\citep{Rus2007}.
The library appears to be unsupported.\footnote{ See \href{https://github.com/JoinPatterns/ScalableJoins}{github.com/JoinPatterns/ScalableJoins}.}

\begin{comment}
(This is not actually JC but CSP!) In 2009, F.~Peschanski published
a JC library for the Lua language, called ``LuaPi''. The library
appears to be unsupported.\footnote{ See \href{https://github.com/fredokun/LuaPi}{github.com/fredokun/LuaPi}.}
\end{comment}

Y.~Liu implemented the basic JC primitives in 2007-2009 as part of
the C++ \texttt{Boost} library.\footnote{ See \href{http://channel.sourceforge.net/}{channel.sourceforge.net}.} 

In 2013, the present author created experimental JC prototypes for
Objective-C on iOS and for Java on Android. The source code is available
from Github\footnote{ See \href{https://github.com/winitzki}{github.com/winitzki}.};
these projects are unsupported. The \texttt{Chymyst} implementation
reuses some design decisions made in these projects.

In 2014, S.~Yallop implemented JoCaml-style ``Join Language'' as
a library in Haskell.\footnote{ See \href{https://github.com/syallop/Join-Language}{github.com/syallop/Join-Language}.}
The implementation uses advanced features of Haskell's type system
to provide a concise syntax for join definitions.

The first embedding of JC as a Scala DSL was P.~Haller's ``Scala
Joins'' library~\citep{HalCut2008}. Thereafter, J.~He improved
upon ``Scala Joins'' by streamlining the syntax, removing restrictions
on pattern matching, and implementing remote processes~\citep{He2014}.
\texttt{Chymyst} is a further development of P.~Haller and J.~He's
syntax for embedding JC into Scala.

C.~Russo's library~\citep{Rus2007} allowed synchronous rendezvous
in join patterns as well as incremental construction of processes
and join definitions, while T.~Rompf's language~\citep{Rom2007}
used a continuation-passing syntax for synchronous channels, instead
of the traditional ``\texttt{reply to}'' syntax. \texttt{Chymyst}
also adopts these design choices.

\section{Programming in \texttt{Chymyst}}

In my experience, the absolute majority of software developers are
unfamiliar with join calculus or its ``channel'' / ``message''
/ ``process'' terminology, but the majority of Scala concurrency
practitioners know about the Actor model. Accordingly, I would argue
that introducing JC concepts to the Scala developer audience should
build upon the Actor model knowledge. However, reasoning about JC
programs is most direct and convenient when using the visual metaphors
and the terminology of the Abstract Chemical Machine (``emitter''
/ ``molecule'' / ``reaction'').

The next subsection is a brief introduction to programming in join
calculus using \texttt{Chymyst}.

\subsection{The chemical metaphor for concurrency}

Neither of the words ``join'' and ``calculus'' are particularly
explanatory or visually suggestive. The ``chemical machine'', on
the other hand, is a visually concrete execution model that can be
directly used for designing and reasoning about a JC program. This
subsection is a brief overview of JC as seen through the chemical
machine metaphor, in order to establish terminology.

To begin, one imagines a \textbf{reaction site}, i.e.~a virtual place
where many molecules are floating around and reacting with each other.
Each molecule has a chemical designation (such as \texttt{a}, \texttt{b},
\texttt{c}) and also \emph{carries a value} of a fixed type. Since
the ``chemistry'' here is completely imaginary, the programmer is
free to declare any number of chemical designations and to choose
the corresponding value types. In \texttt{Chymyst}, these declarations
have the form
\begin{align*}
\text{\texttt{val c }} & \text{\texttt{= m[List[Int]]}}\\
\text{\texttt{val t }} & \text{\texttt{= m[Unit]}}
\end{align*}
and result in creating new \textbf{molecule emitters} \texttt{a} and
\texttt{c}. Emitters can be seen as functions that are called in order
to \textbf{emit} the corresponding molecules into the reaction site:
\begin{align*}
 & \text{\texttt{c(List(1,2,3))}}\\
 & \text{\texttt{t()}}
\end{align*}
The newly emitted molecules will carry the specified values of the
correct types.

Further, the programmer must specify the ``chemical laws'' describing
the permitted reactions between molecules. \texttt{Chymyst} describes
reactions using the syntax of Scala partial functions with a single
\texttt{case} clause, wrapped into an auxiliary method called \texttt{go()}:
\[
\text{\texttt{go \{ case tail(\_) + c(x ::~xs) \ensuremath{\Rightarrow} c(xs) \}}}
\]
This reaction consumes two input molecules, \texttt{tail} and \texttt{c},
and evaluates the reaction body (the Scala code within the \texttt{case}
clause). In this example, the reaction body simply emits one output
molecule, \texttt{c}, with a computed new value \texttt{xs}. In \texttt{Chymyst},
reaction definitions can use all features of Scala partial functions,
including arbitrary guard conditions and pattern matching on molecule
values, as well as arbitrary Scala code within the function body.

Reactions are first-class values:
\[
\text{\texttt{val r1 = go \{ case t(\_) \ensuremath{\Rightarrow} ??? \}}}
\]
Creating the reaction value \texttt{r1} does not schedule any computations;
it merely defines the available computation declaratively. In order
to make the chemical machine run reactions, the programmer needs to
create a reaction site using the \texttt{site()} call, such as \texttt{site(r1,
r2)}. A reaction site typically includes several reactions that may
be declared inline for brevity:
\begin{align*}
\text{\texttt{site( }} & \text{\texttt{go \{ case tail(\_) + c(x ::~xs) \ensuremath{\Rightarrow} c(xs) \},}}\\
 & \text{\text{\texttt{go \{ case c(Nil) }}\texttt{\ensuremath{\Rightarrow} done() \} }}\\
\text{\texttt{) }}
\end{align*}
Once a reaction site is created, the code can emit any number of molecules,
such as \texttt{tail} or \texttt{c}, or molecules bound to other reaction
sites. The chemical machine will interpret the declared ``chemical
laws'' and start reactions whenever appropriate input molecules are
available at each reaction site. According to the operational semantics
of JC, any number of different reactions may start concurrently if
sufficiently many input molecules are available.

The \texttt{Chymyst} project embraces the chemical metaphor and its
visually suggestive terminology. In most academic literature on JC,
molecule emitters are called ``channels'', emitting a molecule with
a value is called ``sending a message on a channel'', blocking molecules
are ``synchronous messages'', reactions are ``processes'', and
reaction sites are ``join definitions''. To make the reading of
the present paper easier for academic researchers, I use the academic
terminology in what follows, except when describing pedagogical approaches
to JC. 

\subsubsection{Synchronous channels as shorthand for continuations{*}{*}{*}}

\subsection{The \texttt{Chymyst} flavor of JC}

In this section, I compare the implementation of JC in \texttt{Chymyst}
to that of JoCaml and motivate the relevant design choices and enhancements.

\subsubsection{``Chemical'' syntax}

The syntax ``\texttt{a(x) + b(y)} $\Rightarrow$ ...'' is reminiscent
of the notation for chemical reactions. It is somewhat easier to read
than the traditional JC syntax ``\texttt{a(x) \& b(y)}''.

Sending a message in \texttt{Chymyst} is implemented as a function
call such as ``\texttt{a(1)}'' and does not require a special keyword
such as JoCaml's ``\texttt{spawn}''.

Synchronous channels have special syntax that resembles continuation-passing:
\[
\text{\texttt{go \{ case a(x) + f(y, cont) \ensuremath{\Rightarrow} cont(x+y) \}}}
\]
The same process is defined in JoCaml as
\[
\text{\texttt{def a(x) \& f(y) = reply x+y to f}}
\]


\subsubsection{Separate channel and process definitions}

JoCaml defines new channels implicitly, as soon as a new join definition
is written. Unfortunately, implicit declaration of new channels is
not possible in \texttt{Chymyst} because Scala macros do not allow
us to insert a new top-level symbol declaration into the code. So,
channel declarations need to be explicit (``\texttt{val a = m{[}Unit{]}}'')
and written separately from process definitions. This is a common
design in embedded DSL implementations of JC.

The separation of channel definitions from process definitions brings
several advantages. One is the ability to create a user-specified
number of new channels and to define processes for them incrementally
(see Sec.~\ref{subsec:First-class-process-definitions} below), removing
the requirement to specify all processes in a join definition at compile
time. Another is an increased code clarity due to the explicit labeling
of blocking vs.~non-blocking channel types, which remains implicit
in JoCaml.

A drawback of this separation is that programmers may create a new
channel but forget to define any processes waiting on that channel.
Sending messages to such ``unbound'' channels is an error that can
only be detected at run time because channels are first-class values.
\texttt{Chymyst} throws an exception in that case.

\subsubsection{Non-linear join patterns}

Another enhancement is the lifting of the linearity restriction for
join patterns. In \texttt{Chymyst}, a process may wait on any number
of repeated channels:
\[
\text{\texttt{go \{ case a(x) + a(y) + a(z) \ensuremath{\Rightarrow} a(x+y+z) \}}}
\]
A linear-pattern JC implementation, such as JoCaml, would require
cumbersome auxiliary definitions to implement this functionality.

Processes may also wait on repeated \emph{synchronous} channels. This
last feature, together with the continuation-passing syntax for those
channels, enables a declarative implementation of a \textbf{synchronous
rendezvous} where two distinct users of the same channel will exchange
values:
\begin{align*}
 & \text{\texttt{go \{ case f(p1, reply1) + f(p2, reply2) \ensuremath{\Rightarrow}}}\\
 & \text{\texttt{ ~ ~ ~ reply2(p1); reply1(p2)}}\\
 & \text{\texttt{\}}}
\end{align*}
JoCaml can express this behavior only at the cost of using two auxiliary
channels and an additional process with a synchronous reply.

\subsubsection{First-class process definitions\label{subsec:First-class-process-definitions}}

Since process definitions in \texttt{Chymyst} are first-class values,
join definitions can be constructed incrementally at run time by aggregating
a dynamically defined number of process definitions. For example,
the well-known ``dining philosophers'' problem has a simple declarative
solution in join calculus (see e.g.~\citep{VarAgh2013}, Sec.~5.4.3)
where the number of philosophers needs to be defined statically. In
\texttt{Chymyst}, this solution can be easily extended to $n$ philosophers
by creating $n$ philosopher and fork channels at run time, defining
an array of $n$ processes for these channels, and aggregating the
$n$ processes into a join definition. 

Despite this, processes and channels remain immutable. Once a join
definition has been created, is impossible to modify its constituent
processes or to add more processes waiting on the same channels.

\subsubsection{Static code analysis}

Scala's macros are used extensively in \texttt{Chymyst} for user code
analysis. The \texttt{go()} macro gathers detailed compile-time information
about input and output channels of each defined process. For example,
the process definition
\[
\text{\texttt{go \{ case t(\_) + c(x ::~xs) \ensuremath{\Rightarrow} c(xs) \}}}
\]
produces a set of flags indicating that the process waits on the channels
\texttt{t} and \texttt{c}, that the channel \texttt{t} may have arbitrary
message values while \texttt{c} requires a pattern match, and that
the process will finally send a message on channel \texttt{c} but
not on \texttt{t}.

Using this information, \texttt{Chymyst} can then detect many cases
of unavoidable livelock, deadlock, and non-determinism in user code.
An example of unavoidable livelock is the process
\[
\text{\texttt{go \{ case c(x) \ensuremath{\Rightarrow} ???; c(x + 1) \}}}
\]
Since the process accepts messages \texttt{c(x)} with any value \texttt{x},
the programmer has no means of stopping the infinite loop that follows
once a single message is sent on channel \texttt{c}. This \texttt{Chymyst}
code generates a \emph{compile-time} error, indicating unavoidable
livelock.

Deadlocks can only happen when using synchronous channels and are
harder to detect reliably. A possible deadlock warning is given when
the process sends a synchronous message followed by another message
that is consumed together with the synchronous one:
\[
\text{\texttt{go \{ case c(x) + f(\_, r) \ensuremath{\Rightarrow} c(f() + 1); r(x) \}}}
\]
This code is suspicious because the process waits for a reply to \texttt{f()}
and \emph{then} sends \texttt{c()}, while a reply to \texttt{f()}
happens only \emph{after} both \texttt{f()} and \texttt{c()} are sent.

Unavoidable non-determinism within a join definition occurs when one
process waits on a subset of messages that another process is waiting
on, for instance
\begin{align*}
\text{\texttt{site( }} & \text{\texttt{go \{ case i(\_) + c(x) \ensuremath{\Rightarrow} c(x + 1) \},}}\\
 & \text{\text{\texttt{go \{ case i(\_) }}\texttt{\ensuremath{\Rightarrow} done() \} }}\\
\text{\texttt{) }}
\end{align*}
If both \texttt{c()} and \texttt{i()} messages are present, the runtime
has a free choice of whether to run the first or the second process.
The programmer has no control over this choice, since there are no
conditions on the values of the \texttt{i()} message. It is unlikely
that the resulting non-determinism would be useful in any practical
application. \texttt{Chymyst} assumes that this is a programmer's
error and throws an exception. The exception is thrown at ``early''
run time, immediately after creating the join definition and before
running any processes.

An additional benefit of static analysis is a performance optimization
for processes that use pattern matching or guard conditions. Scala
macros are used to determine whether process definitions impose any
guard conditions on input message values. If not, a quicker scheduling
algorithm can be used. Additionally, complicated guard conditions
such as
\[
\text{\texttt{go \{ case c(x) + d(y) if x>0 \&\& y<1 \ensuremath{\Rightarrow} ...~\}}}
\]
are converted into the Conjuctive Normal Form and split between molecules
if possible. For instance, the example above is converted to (pseudo-code)
\[
\text{\texttt{\{ case c(x if x>0) + d(y if y<1) \ensuremath{\Rightarrow} ...~\}}}
\]
In many cases, this transformation allows the process scheduler to
select message values faster, without enumerating combinatorially
many combinations of message values while searching for suitable inputs
for a process. 

\subsection{Industry-friendly features{*}{*}{*}}

The number of available concurrent execution threads is implementation-dependent.
\texttt{Chymyst} guarantees that reactions declared at different reaction
sites will be scheduled concurrently, on independent threads.

\subsection{Pedagogical considerations}

The choice of terminology and notation is important if we aim to explain
an unfamiliar paradigm clearly and comprehensibly to newcomers. Here
we again encounter difficulties when it comes to learning about join
calculus. 

The Wikipedia page on JC\footnote{ See \href{https://en.wikipedia.org/wiki/Join-calculus}{en.wikipedia.org/wiki/Join-calculus},
as of December 2016.} describes it as ``\emph{an asynchronous $\pi$-calculus with several
strong restrictions: 1) Scope restriction, reception, and replicated
reception are syntactically merged into a single construct, the }definition\emph{;
2) Communication occurs only on defined names; 3) For every defined
name there is exactly one replicated reception.}''

Explanations using technical jargon such as ``replicated reception''
or ``communication on defined names'' are impenetrable for anyone
not already well-versed in the concurrency research literature. Since
Wikipedia (deservedly or not) is a popular go-to resource for learning
new concepts, it is quite understandable that software practitioners
today remain unaware of join calculus even 20+ years after its invention.

Another obstacle for comprehending JC is that academic literature
typically uses terms such as \textquotedblleft channel\textquotedblright ,
``message'', and ``process'', which are inherited from $\pi$-calculus
but are not helpful for understanding how JC works and how to write
concurrent programs in it.

Indeed, a \textquotedblleft channel\textquotedblright{} in JC holds
an \emph{unordered} collection of messages, rather than an ordered
queue or mailbox, as the word \textquotedblleft channel\textquotedblright{}
suggests. Another meaning of \textquotedblleft channel\textquotedblright{}
is a persistent path for exchanging messages between fixed locations,
but this is far from what a JC \textquotedblleft channel\textquotedblright{}
actually does.

The phrase \textquotedblleft sending a message\textquotedblright{}
usually implies that a fixed recipient will consume the sent messages
one by one. But this is very different from what happens in JC, where
a ``process'' may wait for several \textquotedblleft messages\textquotedblright{}
at once, different ``processes'' may contend on several \textquotedblleft messages\textquotedblright{}
they wait for, and several copies of a ``process'' may start concurrently,
consuming their input ``messages'' in random order.

The word ``process'' suggests a fixed, persistent thread of computation
with which we may communicate. However, JC does not have persistent
threads of computation; instead, ``processes'' are spawned on demand
as input ``messages'' become available.

While JoCaml remains today the only well-maintained standard implementation
of JC, its developer documentation\footnote{ See \href{http://jocaml.inria.fr/doc/index.html}{jocaml.inria.fr/doc/index.html}.}
is especially confusing as regards the semantics of \textquotedblleft channels\textquotedblright ,
\textquotedblleft messages\textquotedblright , \textquotedblleft processes\textquotedblright ,
and ``spawning''. It is ill-suited as a pedagogical introduction
either to using JoCaml or to join calculus. For example, the JoCaml
manual mixes the \texttt{spawn} keyword used for sending messages
with the notion of ``spawning'' a new process, which has a quite
different semantics in JC.

Instead of using academic JC terminology, I follow the chemical machine
metaphor and terminology when giving tutorial presentations about
\texttt{Chymyst} programming. %
\begin{comment}
Here is a dictionary:
\begin{center}
\begin{tabular}{|c|c|c|}
\hline 
Academic join calculus & Chemical machine & \texttt{Chymyst} code example\tabularnewline
\hline 
\hline 
message on a channel & input molecule & \texttt{case a(123)} $\Rightarrow$ ...\tabularnewline
\hline 
channel or name & molecule emitter & \texttt{val a: M{[}Int{]}}\tabularnewline
\hline 
synchronous channel & blocking emitter & \texttt{val q: B{[}Unit, Int{]}}\tabularnewline
\hline 
process & reaction & \texttt{go \{ case a(x) + }...\texttt{ }$\Rightarrow$\texttt{ }...\texttt{ \}}\tabularnewline
\hline 
sending a message & emitting a molecule & \texttt{a(123)}\tabularnewline
\hline 
sending a synchronous message & emitting a blocking molecule & \texttt{val x: Int = q()}\tabularnewline
\hline 
join definition & reaction site & \texttt{site(r1, r2, ...)}\tabularnewline
\hline 
\end{tabular}
\par\end{center}
\end{comment}
With this approach, I have had success in conveying effectively both
the basic concepts and the subtleties of JC semantics to developers
who were previously unfamiliar with it.

\subsection{From actors to molecules}

Many Scala developers interested in concurrent programming are already
familiar with the Actor model. In this subsection, I outline how the
chemical machine paradigm can be introduced to those developers.

In the Actor model, an actor receives messages and reacts to them
by running a computation. An actor-based program declares several
actors, defines the computations for them, stores references to the
actors, and starts sending messages to some of the actors. Messages
are sent either synchronously or asynchronously, enabling communication
between different concurrent actors. 

The chemical machine paradigm is in certain ways similar to the Actor
model. A chemical program also consists of concurrent processes, or
``chemical actors'', that communicate by sending messages. The chemical
machine paradigm departs from the Actor model in two major ways: 
\begin{enumerate}
\item Chemical actors are automatically started and stopped; the user's
code only sends messages and no longer needs to work with actor references.
\item Chemical actors may wait for and consume several messages at once.
\end{enumerate}
If we examine these requirements and determine what should logically
follow from them, we will arrive at the chemical machine paradigm.

The first requirement means that chemical actors are not created explicitly
by the user's program. Instead, the chemical machine runtime will
automatically instantiate and run a chemical actor whenever an input
message is available for consumption. A chemical actor will be automatically
stopped and deleted when its computation is finished. Therefore, the
user's code now does not create an instance of an actor but merely
\emph{defines the computation} that an auto-created actor will perform
after consuming a message. As a consequence, chemical actors must
be \emph{stateless}, and their computations must be functions of the
input message values.

Implementing this functionality will allow us to write pseudo-code
like this,
\begin{align*}
\text{} & \texttt{val c1 }\text{\texttt{= go \{ x:~Int \ensuremath{\Rightarrow} ...~\}}}\\
 & \text{\texttt{c1 !~123 }}
\end{align*}
The computation labeled as \texttt{c1} receives a message with an
\texttt{Int} value and performs some processing on it. The computation
will be instantiated and run concurrently, whenever a message is sent.
In this way, we made the first step towards the full chemical machine
paradigm. 

What should happen if we quickly send many messages? 
\begin{align*}
\text{} & \texttt{val c1 }\text{\texttt{= go \{ x:~Int \ensuremath{\Rightarrow} ...~\}}}\\
 & \text{\texttt{(1 to 100).foreach(c1 !~\_) }}
\end{align*}
Since our computations are stateless, the runtime engine may choose
to run several instances of the computation \texttt{c1} concurrently,
depending on run-time conditions.

Note that \texttt{c1} is not a reference to a particular instance
of a computation. Rather, the computation \texttt{\{ x: Int $\Rightarrow$
... \}} is being defined \emph{declaratively}, as a description of
what needs to be done with any message sent via \texttt{c1}. Thus,
the value \texttt{c1} plays the role of a label attached to the value
\texttt{123} specifying that the value \texttt{123} should be used
as the input value \texttt{x} in the particular computation. To express
this semantics more clearly, let us change our pseudo-code notation
to 
\begin{align*}
\text{} & \text{\texttt{go \{ x:~Int \textnormal{from} c1 \ensuremath{\Rightarrow} ...~\}}}\\
 & \text{\texttt{c1 !~123 }}
\end{align*}
Different chemical actors are now distinguished only by their input
message labels, for example:
\begin{align*}
\text{} & \text{\texttt{go \{ x:~Int \textnormal{from} c1 \ensuremath{\Rightarrow} ...~\}}}\\
\text{} & \text{\texttt{go \{ z:~Int \textnormal{from} e1 \ensuremath{\Rightarrow} ...~\}}}\\
 & \text{\texttt{c1 !~123 }}\\
 & \text{\texttt{e1 !~456 }}
\end{align*}

The second requirement means that a chemical actor should be able
to wait for, say, two messages at once, allowing us to write pseudo-code
like this, 
\begin{align*}
\text{} & \text{\texttt{go \{ x:~Int \textnormal{from} c1, y:~String \textnormal{from} c2 \ensuremath{\Rightarrow} ...~\}}}\\
 & \text{\texttt{c1 !~123}}\\
 & \text{\texttt{c2 !~"abc"}}
\end{align*}
The two messages are of different types and are labeled by \texttt{c1}
and \texttt{c2} respectively. The computation starts only after both
messages have been sent.

It follows that messages cannot be sent to a linearly ordered queue
or a mailbox; instead, messages must be kept in an unordered bag,
as they will be consumed in an unknown order.

It also follows that we may now define several computations that \emph{contend}
on input messages:
\begin{align*}
\text{} & \text{\texttt{go \{ x:~Int \textnormal{from} c1, y:~String \textnormal{from} c2 \ensuremath{\Rightarrow} ...~\}}}\\
 & \text{\texttt{go \{ x:~Int \textnormal{from} c1, z:~Unit \textnormal{from} d1 \ensuremath{\Rightarrow} ...~\}}}
\end{align*}
We find that messages that carry data are now completely decoupled
from computations that consume the data. All computations start concurrently
whenever their input messages become available. The runtime engine
needs to resolve message contention by making a non-deterministic
choice of the messages that will be actually consumed. Among the several
contending computations, only one will be actually started.

This concludes the second and final step towards the chemical machine
paradigm. It remains to use the Scala syntax instead of pseudo-code.
The syntax used by \texttt{Chymyst} looks like this:
\begin{align*}
\text{} & \text{\texttt{val c1 = m[Int]}}\\
 & \text{\texttt{val c2 = m[String]}}\\
 & \text{\texttt{go \{ case c1(x) + c2(y) \ensuremath{\Rightarrow} ...~\}}}\\
 & \text{\texttt{c1(123); }\texttt{c2("abc")}}
\end{align*}

As we have just seen, the chemical machine paradigm is a radical departure
from the Actor model:
\begin{itemize}
\item Whenever there are sufficiently many input messages available for
processing, the runtime engine may automatically instantiate several
concurrent copies of the same computation that will consume the input
messages concurrently. This is the main method for achieving parallelism
in the chemical paradigm. The runtime engine is in the best position
to optimize the CPU load using low-level OS threads. The users do
not need to concern themselves with the details of how many concurrent
actors to instantiate at any given time. 
\item Since chemical actors are stateless and instantiated automatically
on demand, users do not need to implement actor lifecycle management,
actor supervision hierarchies, backup and recovery of internal state
of actors, or a special ``dead letter'' actor. This removes a significant
amount of complexity from the architecture of concurrent applications.
\item Input message contention is used in the chemical machine paradigm
as a general mechanism for synchronization and mutual exclusion. (In
the Actor model, these features are implemented by creating a fixed
number of actor instances that alone can consume certain messages.)
Since the runtime engine will arbitrarily decide which actor to run,
input contention will result in nondeterminism. This is quite similar
to the nondeterminism in the usual models of concurrent programming.
For example, mutual exclusion allows the programmer to implement safe
exclusive access to a resource for any number of concurrent processes,
but the order of access among the contending processes remains unspecified.
\end{itemize}
In the chemical machine paradigm, ``chemical actor'' computations
are called \textbf{reactions}, their input messages are called \textbf{input
molecules}, and messages sent by a chemical computation are called
\textbf{output molecules} of the reaction. 

\section{Future roadmap}

\bibliographystyle{acm}
\bibliography{chymyst_paper}

\end{document}
