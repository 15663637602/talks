%% LyX 2.2.0 created this file.  For more info, see http://www.lyx.org/.
%% Do not edit unless you really know what you are doing.
\documentclass[english]{beamer}
\usepackage[T1]{fontenc}
\usepackage[latin9]{inputenc}
\setcounter{secnumdepth}{3}
\setcounter{tocdepth}{3}
\usepackage{babel}
\ifx\hypersetup\undefined
  \AtBeginDocument{%
    \hypersetup{unicode=true,pdfusetitle,
 bookmarks=true,bookmarksnumbered=false,bookmarksopen=false,
 breaklinks=false,pdfborder={0 0 1},backref=false,colorlinks=true}
  }
\else
  \hypersetup{unicode=true,pdfusetitle,
 bookmarks=true,bookmarksnumbered=false,bookmarksopen=false,
 breaklinks=false,pdfborder={0 0 1},backref=false,colorlinks=true}
\fi

\makeatletter
%%%%%%%%%%%%%%%%%%%%%%%%%%%%%% Textclass specific LaTeX commands.
 % this default might be overridden by plain title style
 \newcommand\makebeamertitle{\frame{\maketitle}}%
 % (ERT) argument for the TOC
 \AtBeginDocument{%
   \let\origtableofcontents=\tableofcontents
   \def\tableofcontents{\@ifnextchar[{\origtableofcontents}{\gobbletableofcontents}}
   \def\gobbletableofcontents#1{\origtableofcontents}
 }

%%%%%%%%%%%%%%%%%%%%%%%%%%%%%% User specified LaTeX commands.
\usetheme[secheader]{Boadilla}
\usecolortheme{seahorse}
\title{Scala \texttt{Thread}s and \texttt{Future}s}
\author{Sergei Winitzki}
\date{September 26, 2017}
\institute[Workday, Inc.]{Workday, Inc.}

\makeatother

\begin{document}
\frame{\titlepage}
\begin{frame}{JVM threads}

Java \texttt{\textcolor{blue}{\small{}Thread}} API: low-level and
underpowered
\begin{itemize}
\item user code must create/maintain threads
\item cannot reuse one thread for running different tasks
\item ``heavy'': cannot create more than about 2,000 JVM threads
\item difficult to synchronize across threads correctly (\texttt{\textcolor{blue}{\small{}wait}}
/ \texttt{\textcolor{blue}{\small{}notify}} / \texttt{\textcolor{blue}{\small{}synchronized}})
\item exceptions on a thread are invisible
\end{itemize}
Plain C multithreading was essentially just as hard
\end{frame}

\begin{frame}{\texttt{ThreadPoolExecutor} = thread pool + task queue}

Java \texttt{\textcolor{blue}{\small{}ThreadPoolExecutor}}:
\begin{itemize}
\item has a queue of pending tasks
\item runs tasks on a dynamically managed thread pool
\item reuses threads for different tasks
\item Fork/Join executor: additional facilities for synchronization
\item cannot be restarted after shutdown
\end{itemize}
Main pattern of usage:
\begin{itemize}
\item Run a large number of short tasks on a small, fixed number of threads
\end{itemize}
\end{frame}

\begin{frame}{Usages of thread pools}

\begin{itemize}
\item \texttt{\textcolor{blue}{\small{}scala.concurrent.ExecutionContext}}
is a wrapper over a thread pool
\item \texttt{\textcolor{blue}{\small{}akka.actor.ActorSystem}} contains
a thread pool
\item Apache's \texttt{\textcolor{blue}{\small{}HttpAsyncClient}} contains
a thread pool
\item Scala's \texttt{\textcolor{blue}{\small{}Future{[}T{]}}} operations
(\texttt{\textcolor{blue}{\small{}Future()}}, \texttt{\textcolor{blue}{\small{}map}},\texttt{\textcolor{blue}{\small{}
flatMap}}) need an implicit \texttt{\textcolor{blue}{\small{}ExecutionContext}}{\small \par}
\end{itemize}
\end{frame}

\begin{frame}{First look at \texttt{scala.concurrent.Future}}

Main features of \texttt{\textcolor{blue}{\small{}scala.concurrent.Future}}:
\begin{itemize}
\item ``value semantics'' \\
\texttt{\textcolor{blue}{\small{}val f:~Future{[}T{]}}} \\
\textendash{} represents a value (of type \texttt{\textcolor{blue}{\small{}T}})
that \emph{will} \emph{become available} in the future
\item ``container semantics'': covariant, has \texttt{\textcolor{blue}{\small{}map}}
and \texttt{\textcolor{blue}{\small{}flatMap}}{\small \par}
\item error handling and failure recovery
\item use \texttt{\textcolor{blue}{\small{}scala.concurrent.Promise{[}T{]}}}
to convert callback APIs to \texttt{\textcolor{blue}{\small{}Future{[}T{]}}}{\small \par}
\end{itemize}
{\footnotesize{}Note: }\texttt{\footnotesize{}java.util.concurrent.Future}{\footnotesize{}
is just a callback wrapper class}{\footnotesize \par}
\end{frame}

\begin{frame}{How does a \texttt{Future} run?}

\texttt{\textcolor{blue}{\small{}implicit ec = somebody.giveMeExecContextPlease()}}{\small \par}

\texttt{\textcolor{blue}{\small{}val f:~Future{[}Array{[}Float{]}{]}
= Future \{ long\_computation() \}}}\textcolor{blue}{\small{}}\\
\texttt{\textcolor{blue}{\small{}logger.info(``Long computation started'')\medskip{}
}}{\small \par}

Getting a \texttt{\textcolor{blue}{\small{}Future{[}T{]}}} value means:
\begin{itemize}
\item a task was queued on that \emph{somebody's} thread pool
\item we could get the result value when it becomes available
\item we have no way of knowing when that will happen
\end{itemize}
\end{frame}

\begin{frame}{How does \texttt{map} work on a \texttt{Future}?}

\texttt{\textcolor{blue}{\small{}implicit ec = somebody.giveMeExecContextPlease()}}{\small \par}

\texttt{\textcolor{blue}{\small{}val f:~Future{[}Array{[}Float{]}{]}
= ...}}\textcolor{blue}{\small{}}\\
\texttt{\textcolor{blue}{\small{}logger.info(``Long computation started'')}}~\\
\texttt{\textcolor{blue}{\small{}val s:~Future{[}Int{]} = f.map(\_.length)}}~\\
{\small \par}
\begin{itemize}
\item using \texttt{map} requires an \texttt{\textcolor{blue}{\small{}ExecutionContext}}{\small \par}
\item the new computation (\texttt{\textcolor{blue}{\small{}\_.length}})
will run once the array is ready
\item the new computation may run on another thread!
\item \texttt{\textcolor{blue}{\small{}flatMap}} also works: \texttt{\textcolor{blue}{\small{}Future{[}Future{[}T{]}{]}}}
can be ``flattened'' to \texttt{\textcolor{blue}{\small{}Future{[}T{]}}}{\small \par}
\end{itemize}
\end{frame}

\begin{frame}{How to work with APIs returning a \texttt{Future}?}

\begin{itemize}
\item get an \texttt{\textcolor{blue}{\small{}ExecutionContext}} (e.g. \texttt{\textcolor{blue}{\small{}actorSystem.dispatcher}})
\item use \texttt{\textcolor{blue}{\small{}map}} or \texttt{\textcolor{blue}{\small{}flatMap}}
to specify computations to be done in the future
\item return another \texttt{\textcolor{blue}{\small{}Future}} value
\item avoid using \texttt{\textcolor{blue}{\small{}Await.result}} if you
can
\end{itemize}
With modern libraries, no need to wait!

\medskip{}

Examples:
\begin{itemize}
\item Akka-Streaming accepts a \texttt{\textcolor{blue}{\small{}Future}}
value inside \texttt{\textcolor{blue}{\small{}mapAsync}}{\small \par}
\item Akka-HTTP accepts a \texttt{\textcolor{blue}{\small{}Future}} value
inside an HTTP route
\end{itemize}
\end{frame}

\begin{frame}{How to convert callback APIs into a \texttt{Future}?}

\begin{itemize}
\item create a\texttt{ }\texttt{\textcolor{blue}{\small{}Promise{[}T{]}}}
value:\\
\texttt{\textcolor{blue}{\small{}val p = Promise{[}Int{]}()}}{\small \par}
\item in the callback's body, resolve the promise:\\
\texttt{\textcolor{blue}{\small{}p.success(123)}}{\small \par}
\item return a \textcolor{blue}{\small{}Future{[}T{]}} value using the promise
value:\\
\texttt{\textcolor{blue}{\small{}p.future }}{\small \par}
\end{itemize}
\end{frame}

\begin{frame}{Example: Converting Apache's \texttt{HttpAsyncClient}}

Converting Apache's \texttt{\textcolor{blue}{\small{}HttpAsyncClient}}
into a \texttt{\textcolor{blue}{\small{}Future}}-based API\medskip{}

\texttt{\textcolor{blue}{\small{}val client:~CloseableHttpAsyncClient
=}}{\small \par}

\texttt{\textcolor{blue}{\small{}~ ~ HttpAsyncClientBuilder.(....).build()}}{\small \par}

\texttt{\textcolor{blue}{\small{}val promise = Promise{[}HttpResponse{]}()}}{\small \par}

\texttt{\textcolor{blue}{\small{}client.execute(request, new FutureCallback{[}HttpResponse{]}
\{}}{\small \par}

\texttt{\textcolor{blue}{\small{}~ override def cancelled():~Unit
=}}{\small \par}

\texttt{\textcolor{blue}{\small{}~ ~ promise.failure(new CancellationException())}}{\small \par}

\texttt{\textcolor{blue}{\small{}~ override def completed(resp:~HttpResponse):~Unit
=}}{\small \par}

\texttt{\textcolor{blue}{\small{}~ ~ promise.success(resp)}}{\small \par}

\texttt{\textcolor{blue}{\small{}~ override def failed(ex:~Exception):~Unit
= promise.failure(ex) }}{\small \par}

\texttt{\textcolor{blue}{\small{}\})}}{\small \par}

\texttt{\textcolor{blue}{\small{}val responseFuture = promise.future}}{\small \par}

\texttt{\textcolor{blue}{\small{}// The new API will return this.}}{\small \par}
\end{frame}

\begin{frame}{``Mistakes were made''}

Typical ``gotchas'' when using \texttt{\textcolor{blue}{\small{}scala.concurrent.Future}}:
\begin{itemize}
\item unnecessarily waiting for futures to complete \textendash{} blocked
threads
\item forgetting to wait for futures to complete \textendash{} race conditions
\item expecting to see a stack trace from exceptions inside a \texttt{\textcolor{blue}{\small{}Future}} 
\item using one thread pool for everything \textendash{} thread starvation
\item forgetting to shut down the thread pool \textendash{} application
never quits 
\end{itemize}
\end{frame}

\begin{frame}{What is ``thread-safe''}

Thread-safe methods:
\begin{itemize}
\item work correctly even if called from many threads in parallel
\item either have no mutable state, or manage it with great care
\end{itemize}
Examples of thread-safe methods: 
\begin{itemize}
\item \texttt{\textcolor{blue}{\small{}AtomicInteger.incrementAndGet()}}{\small \par}
\item \texttt{\textcolor{blue}{\small{}ConcurrentLinkedQueue.add()}}{\small \par}
\end{itemize}
Example of a non-thread-safe method: 
\begin{itemize}
\item \texttt{\textcolor{blue}{\small{}MockitoSugar.mock{[}T{]}()}} \textendash{}
deadlocks when running tests in parallel
\end{itemize}
Workarounds for non-thread-safe code:
\begin{itemize}
\item wrap in \texttt{\textcolor{blue}{\small{}synchronized}}{\small \par}
\item in some cases, can simply use \texttt{\textcolor{blue}{\small{}fork
in Test := true}}{\small \par}
\end{itemize}
\end{frame}

\begin{frame}{What is ``non-blocking''}

Non-blocking methods:
\begin{itemize}
\item return quickly, although they may \emph{schedule} long-running calculations
\item do not perform an idle wait 
\begin{itemize}
\item \texttt{\textcolor{blue}{\small{}Thread.sleep()}} or \texttt{\textcolor{blue}{\small{}Await.result()}}{\small \par}
\end{itemize}
\item do not perform a busy wait 
\begin{itemize}
\item \texttt{\textcolor{blue}{\small{}while( ! isReady() ) \{ doNothingLoop()
\}}}{\small \par}
\end{itemize}
\item do not perform slow I/O (e.g. HTTP with high latency)
\end{itemize}
Thread pools perform best when most tasks are non-blocking calls

\medskip{}

Blocked threads cause suboptimal CPU utilization, a.k.a. ``slowness''
\end{frame}

\begin{frame}{Can we avoid ``blocking''?}

We could avoid blocking if arriving events could \emph{wake up} our
threads... but:
\begin{itemize}
\item A thread cannot be ``woken up'' if it isn't already blocked (``sleeping'')
\item Someone, somewhere has to keep a blocked thread waiting for events
\item However, it does not have to be within \emph{our} thread pools!
\end{itemize}
If we never \texttt{\textcolor{blue}{\small{}Await.result()}}, we
would never get any non-\texttt{\textcolor{blue}{\small{}Future}}
values...
\begin{itemize}
\item Someone, somewhere has to do \texttt{\textcolor{blue}{\small{}Await.result()}}{\small \par}
\item However, it does not have to be within \emph{our} code (except in
unit tests)
\item With the right libraries (Akka, etc.), our code never needs to block
or to do \texttt{\textcolor{blue}{\small{}Await.result()}}{\small \par}
\end{itemize}
\end{frame}

\begin{frame}{How and when to convert ``blocking'' to ``non-blocking''}

Given a 3rd party blocking call \texttt{\textcolor{blue}{\small{}doHttpWork()}},
our options are:
\begin{itemize}
\item Wrap it in a \texttt{\textcolor{blue}{\small{}Future}} using the \texttt{\textcolor{blue}{\small{}scala.concurrent.blocking()}}
function
\item Use a new thread pool dedicated to scheduling \texttt{\textcolor{blue}{\small{}doHttpWork()}}
tasks
\end{itemize}
\medskip{}
Note: It's OK to block if we are \emph{not} within concurrent code
\end{frame}

\begin{frame}{Summary}

\begin{itemize}
\item Sample code: \href{https://github.com/winitzki/scala-threads-futures-intro}{https://github.com/winitzki/scala-threads-futures-intro}
\item The backbone of Java concurrency: thread pool executors
\item How and when do \texttt{\textcolor{blue}{\small{}Future}}s run? On
thread pools
\item What does it mean to be \textquotedblleft thread safe\textquotedblright{}
and \textquotedblleft nonblocking\textquotedblright ?
\begin{itemize}
\item nonblocking = \textcolor{blue}{\small{}uses} multi-core CPU optimally
\item thread safe = permits easy concurrent nonblocking code
\end{itemize}
\item Some typical \textquotedblleft gotchas\textquotedblright{} when using
\texttt{\textcolor{blue}{\small{}Future}}s in the real world
\item Converting other async APIs to \texttt{\textcolor{blue}{\small{}Future}}s
and back
\end{itemize}
\end{frame}

\end{document}
